\chapter{RÉFÉRENCES}

\subsection{Documentation officielle}

\subsubsection{Google ADK et technologies associées}
\begin{itemize}
    \item Google. (2024). \textit{Agent Development Kit Documentation}. \url{https://google.github.io/adk-docs/}
    \item Google. (2024). \textit{Gemini AI Model Documentation}. \url{https://ai.google.dev/gemini-api}
    \item Google Cloud. (2024). \textit{Vertex AI Documentation}. \url{https://cloud.google.com/vertex-ai/docs}
    \item Python Software Foundation. (2024). \textit{Python 3.12 Documentation}. \url{https://docs.python.org/3.12/}
    \item Poetry. (2024). \textit{Poetry Dependency Management}. \url{https://python-poetry.org/docs/}
\end{itemize}

\subsubsection{Standards et protocoles}
\begin{itemize}
    \item FIPA. (2002). \textit{FIPA Agent Communication Language Specifications}. Foundation for Intelligent Physical Agents
    \item W3C. (2024). \textit{Semantic Web Standards}. \url{https://www.w3.org/standards/semanticweb/}
    \item OGC. (2024). \textit{Open Geospatial Consortium Standards}. \url{https://www.ogc.org/standards}
\end{itemize}

\subsection{Articles et publications}

\subsubsection{Systèmes multi-agents}
\begin{itemize}
    \item Wooldridge, M. (2009). \textit{An Introduction to MultiAgent Systems}. 2nd Edition, John Wiley \& Sons
    \item Stone, P., \& Veloso, M. (2000). "Multiagent Systems: A Survey from a Machine Learning Perspective". \textit{Autonomous Robots}, 8(3), 345-383
    \item Jennings, N. R. (2001). "An agent-based approach for building complex software systems". \textit{Communications of the ACM}, 44(4), 35-41
\end{itemize}

\subsubsection{Intelligence artificielle en agriculture}
\begin{itemize}
    \item Liakos, K. G., et al. (2018). "Machine learning in agriculture: A review". \textit{Sensors}, 18(8), 2674
    \item Kamilaris, A., \& Prenafeta-Boldú, F. X. (2018). "Deep learning in agriculture: A survey". \textit{Computers and Electronics in Agriculture}, 147, 70-90
    \item Wolfert, S., et al. (2017). "Big data in smart farming–a review". \textit{Agricultural systems}, 153, 69-80
\end{itemize}

\subsubsection{Agriculture africaine et technologie}
\begin{itemize}
    \item Aker, J. C. (2011). "Dial "A" for agriculture: a review of information and communication technologies for agricultural extension in developing countries". \textit{Agricultural economics}, 42(6), 631-647
    \item Trendov, N. M., et al. (2019). \textit{Digital technologies in agriculture and rural areas}. FAO
    \item Totin, E., et al. (2018). "Institutional perspectives of climate-smart agriculture: A systematic literature review". \textit{Sustainability}, 10(6), 1990
\end{itemize}

\subsection{Ressources complémentaires}

\subsubsection{Bases de données et APIs}
\begin{itemize}
    \item OpenWeatherMap API : \url{https://openweathermap.org/api}
    \item NASA Earth Data : \url{https://earthdata.nasa.gov/}
    \item FAO GIEWS : \url{http://www.fao.org/giews/en/}
    \item World Bank Climate Data : \url{https://climateknowledgeportal.worldbank.org/}
\end{itemize}

\subsubsection{Outils et frameworks complémentaires}
\begin{itemize}
    \item MESA : Agent-based modeling framework. \url{https://mesa.readthedocs.io/}
    \item NetLogo : Multi-agent programmable modeling environment. \url{https://ccl.northwestern.edu/netlogo/}
    \item SUMO : Multi-agent traffic simulation. \url{https://www.eclipse.org/sumo/}
    \item Pandas : Data manipulation library. \url{https://pandas.pydata.org/}
    \item FastAPI : Modern web framework for APIs. \url{https://fastapi.tiangolo.com/}
\end{itemize}
