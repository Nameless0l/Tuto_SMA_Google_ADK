\chapter{Annexe}
\section{Annexe A : Glossaire des termes SMA}
\begin{description}
    \item[Agent] Entité autonome capable de percevoir son environnement et d'agir de manière indépendante pour atteindre ses objectifs.

    \item[Autonomie] Capacité d'un agent à prendre des décisions et à agir sans intervention externe directe.

    \item[Comportement (Behavior)] Ensemble d'actions et de réactions d'un agent face aux stimuli de son environnement.

    \item[Communication inter-agents] Mécanisme permettant aux agents d'échanger des informations et de coordonner leurs actions.

    \item[Coordination] Processus par lequel les agents synchronisent leurs actions pour atteindre un objectif commun.

    \item[Émergence] Phénomène par lequel des propriétés complexes apparaissent au niveau système à partir d'interactions simples entre agents.

    \item[Environnement] Contexte dans lequel évoluent les agents, incluant les ressources et les contraintes.

    \item[LLM (Large Language Model)] Modèle d'intelligence artificielle capable de comprendre et générer du langage naturel.

    \item[Multi-agent] Système composé de plusieurs agents interagissant dans un environnement partagé.

    \item[Ontologie] Représentation formelle des connaissances d'un domaine spécifique.

    \item[Performative] Type d'acte de communication dans le langage ACL (ex: INFORM, REQUEST, PROPOSE).

    \item[Pro-activité] Capacité d'un agent à prendre des initiatives et à anticiper les besoins.

    \item[Réactivité] Capacité d'un agent à répondre rapidement aux changements de son environnement.

    \item[Socialité] Capacité d'un agent à interagir et collaborer avec d'autres agents.

    \item[Système expert] Système d'intelligence artificielle simulant le raisonnement d'un expert humain dans un domaine spécifique.
\end{description}


\section{Annexe B : Commandes utiles et dépannage}

\subsection{Installation et configuration}
\begin{verbatim}
# Installation de Python 3.12+
sudo apt update
sudo apt install python3.12 python3.12-pip

# Vérification de la version
python3.12 --version

# Installation de Poetry
curl -sSL https://install.python-poetry.org | python3 -

# Configuration du projet
poetry new agriculture_cameroun
cd agriculture_cameroun
poetry add google-adk aiohttp fastapi uvicorn

# Variables d'environnement
export GOOGLE_API_KEY="votre_clé_ici"
export OPENWEATHER_API_KEY="votre_clé_ici"

# Lancement du système
poetry run python src/main.py

# Tests
poetry run pytest tests/

# Interface web
poetry run uvicorn src.api:app --reload --port 8000
\end{verbatim}

\subsection{Debugging et logs}
\begin{verbatim}
# Activation du mode debug
export DEBUG=True

# Logs détaillés
export LOG_LEVEL=DEBUG

# Monitoring des performances
poetry add psutil
python -c "
import psutil
print(f'CPU: {psutil.cpu_percent()}%')
print(f'RAM: {psutil.virtual_memory().percent}%')
"

# Test de connectivité API
curl -X GET "http://api.openweathermap.org/data/2.5/weather?q=Yaoundé&appid=YOUR_KEY"

# Validation du format JSON
python -m json.tool data/crops_cameroon.json

# Profiling du code
poetry add py-spy
py-spy record -o profile.svg -- python src/main.py
\end{verbatim}

\subsection{Maintenance et mise à jour}
\begin{verbatim}
# Mise à jour des dépendances
poetry update

# Sauvegarde de la base de données
cp -r data/ backup/data_$(date +%Y%m%d_%H%M%S)/

# Nettoyage des logs
find logs/ -name "*.log" -mtime +30 -delete

# Test de santé du système
poetry run python tests/health_check.py

# Monitoring continu
watch -n 10 'curl -s http://localhost:8000/health | jq .'
\end{verbatim}

\section{Annexe C : FAQ et problèmes courants}

\subsection{Questions fréquentes}

\textbf{Q1 : Comment ajouter un nouvel agent spécialisé ?}

R : Pour ajouter un nouvel agent, suivez ces étapes :
\begin{enumerate}
    \item Créez une nouvelle classe héritant de \texttt{Agent}
    \item Définissez les outils spécifiques avec le décorateur \texttt{@tool}
    \item Ajoutez l'agent au dictionnaire \texttt{sub\_agents} de l'agent principal
    \item Mettez à jour la méthode \texttt{\_analyze\_query} avec les mots-clés appropriés
    \item Ajoutez des tests unitaires dans \texttt{tests/}
\end{enumerate}

\textbf{Q2 : Comment intégrer une nouvelle API externe ?}

R : Créez un nouveau module dans \texttt{src/tools/} avec :
\begin{itemize}
    \item Une classe d'interface API
    \item Gestion des erreurs et retry logic
    \item Cache pour optimiser les performances
    \item Tests de l'intégration
\end{itemize}

\textbf{Q3 : Comment personnaliser les recommandations par région ?}

R : Modifiez le fichier \texttt{agriculture\_cameroun/utils/data.py} :
\begin{itemize}
    \item Ajoutez des données spécifiques par région
    \item Adaptez les algorithmes de recommandation
    \item Mettez à jour la base de connaissances régionale
\end{itemize}

\subsection{Problèmes courants et solutions}

\textbf{Erreur : "API Key non valide"}
\begin{itemize}
    \item Vérifiez que les clés API sont correctement définies dans les variables d'environnement
    \item Testez les clés avec un appel direct à l'API
    \item Régénérez les clés si nécessaire
\end{itemize}

\textbf{Erreur : "TimeoutError lors des appels API"}
\begin{itemize}
    \item Augmentez le timeout dans la configuration
    \item Implémentez un mécanisme de retry
    \item Utilisez un cache pour réduire les appels
\end{itemize}


\textbf{Erreur : "Module non trouvé"}
\begin{itemize}
    \item Vérifiez que Poetry est correctement installé
    \item Exécutez \texttt{poetry install} pour installer les dépendances
    \item Activez l'environnement virtuel avec \texttt{poetry shell}
\end{itemize}

\textbf{Interface web ne se lance pas}
\begin{itemize}
    \item Vérifiez que le port 8000 n'est pas déjà utilisé
    \item Contrôlez les logs pour identifier l'erreur
    \item Testez avec un port différent : \texttt{--port 8001}
\end{itemize}

\subsection{Conseils de performance}

\begin{itemize}
    \item \textbf{Cache intelligent} : Implémentez un cache hiérarchique pour les données météo et market
    \item \textbf{Parallélisation} : Utilisez \texttt{asyncio.gather()} pour les appels simultanés aux agents
    \item \textbf{Pagination} : Limitez le nombre de résultats retournés par les APIs
    \item \textbf{Compression} : Compressez les réponses HTTP avec gzip
    \item \textbf{Monitoring} : Utilisez des métriques pour identifier les goulots d'étranglement
\end{itemize}

\subsection{Ressources de support}

\begin{itemize}
    \item \textbf{Documentation officielle} : \url{https://github.com/Nameles0l/agriculture-cameroun/}
    \item \textbf{Issues GitHub} : \url{https://github.com/Nameles0l/agriculture-cameroun/issues}
    \item \textbf{Email support} : --
\end{itemize}
