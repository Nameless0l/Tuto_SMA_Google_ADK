\chapter*{Introduction}

Dans ce tutoriel, nous allons explorer le développement de \textbf{systèmes multi-agents (SMA)} à travers la création d'un système d'assistance agricole pour le Cameroun. Nous utiliserons \textbf{Google Agent Development Kit (ADK)}, un framework moderne qui permet de créer des agents intelligents capables de collaborer pour résoudre des problèmes complexes. Ce document vous guidera pas à pas depuis les concepts fondamentaux jusqu'à l'implémentation complète d'un système fonctionnel.

\section*{Objectifs du tutoriel}

Ce tutoriel vise à vous fournir une compréhension approfondie des \textbf{systèmes multi-agents} et la maîtrise pratique de \textbf{Google ADK}. Vous apprendrez d'abord les \emph{concepts fondamentaux} qui sous-tendent les SMA, notamment les notions d'\textbf{agent}, d'\textbf{autonomie}, de \textbf{communication inter-agents}, de \textbf{protocoles d'interaction} et d'\textbf{ontologies}. Cette base théorique solide vous permettra de comprendre comment les agents peuvent collaborer efficacement pour résoudre des problèmes complexes.

Vous découvrirez ensuite \textbf{Google ADK}, un framework moderne qui révolutionne le développement d'agents intelligents grâce à son intégration native avec les \emph{modèles de langage}. Pour ceux ayant déjà une expérience avec \textbf{JADE}, nous établirons des parallèles et soulignerons les différences majeures entre ces deux approches, facilitant ainsi la transition vers ce nouveau paradigme.

L'objectif principal reste l'\emph{implémentation pratique} d'un système multi-agents complet pour l'agriculture camerounaise. Vous construirez progressivement \textbf{cinq agents spécialisés} (\emph{Météorologique}, \emph{Cultures}, \emph{Santé des Plantes}, \emph{Économique} et \emph{Ressources}), coordonnés par un \textbf{agent principal}. Cette approche pratique vous permettra de maîtriser les mécanismes de communication inter-agents, les protocoles d'interaction et les stratégies de coordination. Enfin, vous apprendrez à \textbf{déployer} et \textbf{tester} votre système dans un environnement réel.

\section*{Public cible et prérequis}

Ce tutoriel s'adresse principalement aux \textbf{étudiants en informatique} suivant un cours sur les systèmes multi-agents, mais également aux \textbf{développeurs} souhaitant découvrir cette technologie et aux \textbf{professionnels du secteur agricole} intéressés par les solutions intelligentes. La progression pédagogique a été conçue pour accompagner différents niveaux d'expertise, depuis les concepts de base jusqu'aux implémentations avancées.

Pour tirer pleinement profit de ce tutoriel, vous devez posséder une connaissance solide du langage \textbf{Python}, incluant la \emph{programmation orientée objet}, la gestion des \emph{modules} et \emph{packages}. Une compréhension des concepts de base en \textbf{intelligence artificielle}, notamment les notions d'agents intelligents et de systèmes distribués, facilitera grandement votre apprentissage. L'expérience avec un \textbf{environnement de développement intégré} comme \emph{VS Code} et la familiarité avec les outils en \emph{ligne de commande} sont également nécessaires. Des notions de base de \textbf{Git} vous permettront de cloner le dépôt du projet et de suivre les exemples de code.

Sur le plan matériel, assurez-vous de disposer d'un ordinateur sous \emph{Windows 10/11}, \emph{macOS 10.15+} ou \emph{Linux Ubuntu 20.04+}, avec \textbf{Python 3.12} ou une version supérieure installée. Un minimum de \textbf{8 GB de RAM} est requis, bien que 16 GB soient recommandés pour une expérience optimale. Prévoyez environ \textbf{2 GB d'espace disque} disponible et une \emph{connexion Internet stable} pour les téléchargements et l'accès aux API externes.

\section*{Vue d'ensemble du projet Agriculture Cameroun}

Le projet \textbf{Agriculture Cameroun} représente une réponse innovante aux défis complexes du secteur agricole camerounais. Dans un contexte où les agriculteurs font face à une \emph{variabilité climatique} croissante, des \emph{maladies des cultures} imprévisibles, des \emph{fluctuations des prix} du marché et une gestion souvent sub-optimale des \emph{ressources limitées}, l'accès à une information fiable et personnalisée devient crucial pour la prise de décision.

Les agriculteurs camerounais rencontrent quotidiennement des \textbf{obstacles majeurs} dans leur activité. L'accès aux \emph{informations météorologiques} fiables et localisées reste limité, rendant difficile la planification des activités agricoles. Le \emph{diagnostic des maladies} des plantes et le choix des \emph{traitements appropriés} constituent un défi constant, souvent aggravé par le manque d'expertise technique disponible localement. Les informations sur les \emph{prix du marché} et la \emph{rentabilité} des différentes cultures sont fragmentées et peu accessibles, compliquant les décisions économiques. La gestion des ressources précieuses comme l'\textbf{eau}, les \textbf{engrais} et les \textbf{semences} se fait souvent de manière empirique, sans optimisation réelle. Enfin, l'absence de \emph{conseils personnalisés} adaptés au contexte spécifique de chaque exploitation limite le potentiel de productivité.

Face à ces défis, notre système multi-agents propose une \textbf{plateforme intelligente} où cinq agents spécialisés collaborent harmonieusement. L'\textbf{Agent Météorologique} collecte et analyse les données climatiques pour fournir des prévisions localisées et des alertes pertinentes. L'\textbf{Agent Cultures} s'appuie sur une base de connaissances agronomiques pour conseiller sur les pratiques culturales optimales et les périodes de semis. L'\textbf{Agent Santé des Plantes} utilise des techniques de reconnaissance et d'analyse pour diagnostiquer les problèmes phytosanitaires et proposer des traitements adaptés. L'\textbf{Agent Économique} analyse les tendances du marché et aide à évaluer la rentabilité des différentes options culturales. L'\textbf{Agent Ressources} optimise l'utilisation des intrants agricoles en proposant des stratégies de gestion durable.

Ces agents ne fonctionnent pas en isolation mais sont orchestrés par un \textbf{agent coordinateur principal} qui joue un rôle crucial dans le système. Cet agent reçoit les requêtes des utilisateurs formulées en \emph{langage naturel}, analyse leur intention pour déterminer quels agents spécialisés solliciter, coordonne les interactions entre agents pour les requêtes complexes, et synthétise les différentes réponses pour fournir une information cohérente et directement actionnable par l'agriculteur.

L'utilisation de \textbf{Google ADK} comme framework de développement apporte une dimension moderne au projet. L'intégration native avec les \emph{modèles de langage Gemini} permet des interactions naturelles et intuitives avec les utilisateurs. La capacité d'analyser et de traiter des \emph{données complexes} provenant de sources multiples enrichit considérablement la qualité des recommandations. L'architecture flexible d'ADK facilite l'ajout de nouveaux agents ou l'extension des capacités existantes selon l'évolution des besoins.