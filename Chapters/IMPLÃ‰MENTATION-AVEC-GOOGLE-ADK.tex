\chapter{IMPLÉMENTATION AVEC GOOGLE ADK}

\section{Concepts de Base ADK}

\subsection{Création d'un agent simple}

La création d'un agent avec Google ADK représente un changement de paradigme par rapport aux frameworks traditionnels. Au lieu de programmer explicitement chaque comportement, nous définissons les capacités et objectifs de l'agent, laissant le modèle de langage générer les comportements appropriés. Commençons par créer un agent météorologique simple pour illustrer les concepts fondamentaux.

 
\begin{lstlisting}[language=Python, caption=Agent météorologique avec ADK]
# sub_agents/weather/agent.py
import os
from google.adk.agents import Agent
from google.genai import types

from .prompts import return_instructions_weather
from .tools import (
    get_weather_forecast,
    get_irrigation_advice,
    get_climate_alerts,
    analyze_rainfall_patterns
)

weather_agent = Agent(
    model=os.getenv("WEATHER_AGENT_MODEL"),
    name="weather_agent",
    instruction=return_instructions_weather(),
    tools=[
        get_weather_forecast,
        get_irrigation_advice,
        get_climate_alerts,
        analyze_rainfall_patterns
    ],
    generate_content_config=types.GenerateContentConfig(temperature=0.5),
)
\end{lstlisting}

Ce code illustre les concepts fondamentaux d'ADK. L'objet \texttt{Agent} encapsule toute la logique nécessaire pour créer un agent intelligent. Le paramètre \texttt{name} fournit une identité unique à l'agent, essentielle pour la communication inter-agents. Le \texttt{model} spécifie la version de Gemini à utiliser depuis les variables d'environnement. L'\texttt{instruction} définit le comportement de l'agent en langage naturel, une approche radicalement différente de la programmation traditionnelle.

Les \textbf{outils (tools)} représentent l'interface entre l'agent et le monde extérieur. Chaque fonction Python devient un outil utilisable par l'agent. ADK analyse automatiquement la signature de la fonction et sa docstring pour comprendre quand et comment l'utiliser.

\subsection{Cycle de vie d'un agent ADK}

Le cycle de vie d'un agent ADK diffère significativement des agents traditionnels, intégrant de manière transparente les capacités des modèles de langage dans chaque phase d'exécution.

 \begin{figure}[H]
\centering
\framebox[0.9\textwidth]{
\parbox{0.85\textwidth}{
\centering
\textbf{Cycle de vie d'un agent ADK}\\[10pt]
\begin{enumerate}
\item \textbf{Initialisation} : Chargement du modèle et des instructions
\item \textbf{Réception} : Traitement de la requête utilisateur
\item \textbf{Analyse} : Compréhension via le LLM
\item \textbf{Planification} : Détermination des actions nécessaires
\item \textbf{Exécution} : Invocation des outils si nécessaire
\item \textbf{Synthèse} : Génération de la réponse
\item \textbf{Retour} : Transmission du résultat
\end{enumerate}
}
}
\caption{Les phases du cycle de vie d'un agent ADK}
\end{figure}

Durant la phase d'\textbf{initialisation}, l'agent charge ses instructions depuis le fichier \texttt{prompts.py} et configure sa connexion avec le modèle Gemini. Cette phase inclut la validation des outils disponibles et la préparation du contexte initial. Contrairement aux frameworks traditionnels où l'initialisation implique le chargement de règles complexes, ADK se contente de préparer l'instruction système qui guidera le comportement de l'agent.

La phase de \textbf{réception et analyse} exploite pleinement les capacités de compréhension du langage naturel de Gemini. L'agent n'a pas besoin de parser explicitement la requête ou de la faire correspondre à des patterns prédéfinis. Le modèle comprend l'intention, le contexte et les nuances de la requête, permettant une interaction beaucoup plus naturelle et flexible.

 
\begin{lstlisting}[language=Python, caption=Gestion du contexte avec callback]
# agriculture_cameroun/agent.py
def setup_before_agent_call(callback_context: CallbackContext):
    """Configuration avant l'appel de l'agent."""
    
    # Configuration de l'état de session
    if "agriculture_settings" not in callback_context.state:
        agriculture_settings = {
            "regions": REGIONS,
            "crops": CROPS,
            "seasons": SEASONS,
            "current_date": date_today,
            "default_region": os.getenv("DEFAULT_REGION", "Centre"),
            "language": os.getenv("DEFAULT_LANGUAGE", "fr")
        }
        callback_context.state["agriculture_settings"] = agriculture_settings
    
    # Mise à jour des instructions avec le contexte
    context = callback_context.state["agriculture_settings"]
    callback_context._invocation_context.agent.instruction = (
        return_instructions_root()
        + f"""
        
        Contexte actuel:
        - Date: {context['current_date']}
        - Région par défaut: {context['default_region']}
        - Cultures principales: {', '.join(context['crops'].keys())}
        - Régions disponibles: {', '.join(context['regions'].keys())}
        """
    )
\end{lstlisting}

La phase de \textbf{planification} représente l'intelligence de l'agent en action. Le modèle détermine automatiquement quels outils utiliser, dans quel ordre, et comment combiner leurs résultats. Cette planification implicite élimine le besoin de définir des arbres de décision complexes ou des machines à états.

L'\textbf{exécution} des outils se fait de manière transparente. ADK gère automatiquement la sérialisation des paramètres, l'appel de fonction, la gestion des erreurs et la désérialisation des résultats. L'agent peut décider d'appeler plusieurs outils en séquence ou en parallèle selon les besoins, optimisant automatiquement le flux d'exécution.

\subsection{Gestion des comportements}

Dans ADK, les comportements ne sont pas programmés explicitement mais émergent de la combinaison des instructions, du contexte et des capacités du modèle. Cette approche offre une flexibilité sans précédent tout en maintenant un contrôle sur les actions de l'agent.

\begin{lstlisting}[language=Python, caption=Instructions pour comportements adaptatifs]
# agriculture_cameroun/prompts.py
def return_instructions_root() -> str:
    """Retourne les instructions pour l'agent principal."""
    
    instruction_prompt = """
    Tu es un expert agricole senior chargé de coordonner un système multi-agents pour l'agriculture au Cameroun.
    Ton rôle est d'analyser les demandes des agriculteurs et de diriger les questions vers les agents spécialisés appropriés.
    
    ## Agents disponibles:
    
    1. **Agent Météo** (`call_weather_agent`): Pour toutes les questions concernant:
       - Prévisions météorologiques
       - Conseils d'irrigation
       - Alertes climatiques
       - Conditions météo pour les cultures
    
    2. **Agent Cultures** (`call_crops_agent`): Pour les questions sur:
       - Calendriers de plantation
       - Rotation des cultures
       - Variétés recommandées
       - Techniques de culture
    
    3. **Agent Santé des Plantes** (`call_health_agent`): Pour:
       - Diagnostic de maladies
       - Identification de parasites
       - Traitements recommandés
       - Mesures préventives
    
    4. **Agent Économique** (`call_economic_agent`): Pour:
       - Prix du marché
       - Analyse de rentabilité
       - Conseils de vente
       - Opportunités commerciales
    
    5. **Agent Ressources** (`call_resources_agent`): Pour:
       - Gestion du sol
       - Recommandations d'engrais
       - Irrigation efficace
       - Conservation des ressources
    
    ## Workflow:
    
    1. **Comprendre la demande**: Analyse attentivement la question de l'agriculteur
    2. **Identifier les agents nécessaires**: Détermine quel(s) agent(s) peuvent répondre
    3. **Appeler les agents**: Utilise les outils appropriés avec des questions précises
    4. **Synthétiser les réponses**: Combine les informations reçues
    5. **Répondre à l'agriculteur**: Fournis une réponse claire et pratique
    
    ## Règles importantes:
    
    - Toujours répondre en français
    - Adapter les conseils au contexte camerounais
    - Privilégier les solutions locales et économiques
    - Inclure les coûts estimés en FCFA quand pertinent
    - Mentionner les pratiques traditionnelles quand approprié
    - Si plusieurs agents sont nécessaires, les appeler dans l'ordre logique
    - Ne jamais inventer d'informations, toujours utiliser les agents
    """
    
    return instruction_prompt
\end{lstlisting}

Les \textbf{comportements contextuels} permettent à l'agent d'adapter automatiquement ses réponses selon la situation. L'agent analyse non seulement le contenu explicite de la requête mais aussi le contexte implicite, l'urgence perçue et l'historique de la conversation pour choisir le comportement approprié.

La \textbf{composition de comportements} permet de créer des agents sophistiqués sans complexité excessive. Au lieu de définir des hiérarchies de comportements complexes comme dans JADE, ADK permet de décrire les comportements souhaités en langage naturel, laissant le modèle orchestrer leur activation.

\subsection{Système de prompts et instructions}

Le système de prompts constitue l'âme d'un agent ADK, définissant sa personnalité, ses connaissances et ses patterns de comportement. La maîtrise de l'ingénierie des prompts est essentielle pour créer des agents efficaces et fiables.

 
\begin{lstlisting}[language=Python, caption=Instructions spécialisées pour l'agent santé]
# sub_agents/health/prompts.py
def return_instructions_health() -> str:
    """Retourne les instructions pour l'agent santé des plantes."""
    
    instruction_prompt = """
    Tu es un phytopathologiste expert spécialisé dans la santé des cultures camerounaises.
    Ton rôle est de diagnostiquer les maladies, identifier les parasites et recommander des traitements adaptés au contexte local.
    
    ## Capacités principales:
    
    1. **Diagnostic des maladies**: Identification des pathogènes fongiques, bactériens et viraux
    2. **Identification des parasites**: Reconnaître les insectes nuisibles et ravageurs
    3. **Recommandations de traitement**: Solutions curatives et préventives
    4. **Gestion intégrée**: Approches combinant méthodes biologiques, culturales et chimiques
    5. **Prévention**: Stratégies pour éviter les problèmes sanitaires
    
    ## Outils disponibles:
    
    - `diagnose_plant_disease`: Diagnostic de maladies basé sur les symptômes
    - `get_treatment_recommendations`: Recommandations de traitement spécifiques
    - `get_pest_identification`: Identification des parasites et ravageurs
    - `get_prevention_strategies`: Stratégies de prévention personnalisées
    
    ## Contexte phytosanitaire camerounais:
    
    ### Maladies principales par culture:
    
    **Cacao:**
    - Pourriture brune (Phytophthora palmivora)
    - Mirides (Sahlbergella singularis)
    - Chancre du cacaoyer (Phytophthora megakarya)
    - Maladie du balai de sorcière (Moniliophthora perniciosa)
    
    **Café:**
    - Rouille orangée (Hemileia vastatrix)
    - Anthracnose (Colletotrichum kahawae)
    - Scolytes (Hypothenemus hampei)
    
    **Maïs:**
    - Charbon du maïs (Ustilago maydis)
    - Striure du maïs (Maize streak virus)
    - Foreurs de tige (Sesamia calamistis)
    """
    
    return instruction_prompt
\end{lstlisting}

La \textbf{structure des instructions} suit des patterns éprouvés pour maximiser l'efficacité. L'identité et le rôle établissent le contexte général. Les domaines d'expertise délimitent les connaissances de l'agent. Les contraintes définissent les garde-fous comportementaux. Cette structure guide le modèle tout en laissant la flexibilité nécessaire pour des réponses naturelles.

\section{Communication Inter-Agents}

\subsection{Mécanisme de communication dans ADK}

La communication inter-agents dans ADK utilise le pattern \texttt{AgentTool} qui permet des échanges structurés entre agents tout en conservant la flexibilité du langage naturel.

 
\begin{lstlisting}[language=Python, caption=Communication inter-agents avec AgentTool]
# agriculture_cameroun/tools.py
from google.adk.tools import ToolContext
from google.adk.tools.agent_tool import AgentTool
from typing import Optional

from .sub_agents import (
    weather_agent,
    crops_agent,
    health_agent,
    economic_agent,
    resources_agent
)

async def call_weather_agent(
    question: str,
    tool_context: ToolContext,
    region: Optional[str] = None,
):
    """Appelle l'agent météo pour les questions climatiques.
    
    Args:
        question: Question sur la météo ou le climat
        region: Région spécifique (optionnel)
        tool_context: Contexte de l'outil
    
    Returns:
        Réponse de l'agent météo
    """
    if region is None:
        region = tool_context.state["agriculture_settings"]["default_region"]
    
    agent_tool = AgentTool(agent=weather_agent)
    
    weather_input = {
        "request": question,
        "region": region
    }
    
    response = await agent_tool.run_async(
        args=weather_input,
        tool_context=tool_context
    )
    
    tool_context.state["weather_response"] = response
    return response
\end{lstlisting}

Le \textbf{système de communication ADK} adopte une approche qui combine la structure nécessaire pour la fiabilité avec la flexibilité du langage naturel. Chaque appel d'agent utilise \texttt{AgentTool} avec des paramètres structurés tout en permettant un contenu en langage naturel que les agents peuvent interpréter selon leur contexte et expertise.

La \textbf{gestion asynchrone} des appels permet aux agents de traiter les requêtes de manière efficace. Cette approche évite les blocages et permet un traitement parallèle efficace des requêtes complexes nécessitant l'intervention de plusieurs agents.

\subsection{Implémentation des outils (tools)}

Les outils dans ADK représentent le pont entre l'intelligence linguistique des agents et les actions concrètes dans le monde réel. Leur implémentation correcte est cruciale pour créer des agents véritablement utiles.

 
\begin{lstlisting}[language=Python, caption=Outil de diagnostic des maladies]
# sub_agents/health/tools.py
from typing import Dict, List, Any, Optional
import google.generativeai as genai
from google.adk.tools import ToolContext

def diagnose_plant_disease(
    crop: str,
    symptoms: List[str],
    tool_context: ToolContext,
    affected_parts: Optional[List[str]] = None,
    environmental_conditions: Optional[str] = None,
) -> Dict[str, Any]:
    """Diagnostique une maladie des plantes basée sur les symptômes.
    
    Args:
        crop: Type de culture affectée
        symptoms: Liste des symptômes observés
        affected_parts: Parties de la plante affectées (optionnel)
        environmental_conditions: Conditions environnementales (optionnel)
        tool_context: Contexte de l'outil
        
    Returns:
        Diagnostic détaillé avec probabilités
    """
    # Base de données des maladies étendues
    disease_database = {
        "cacao": [
            {
                "name": "Pourriture brune",
                "agent": "Phytophthora palmivora",
                "symptoms": ["taches brunes", "pourriture fruits", "brunissement cabosses", "exsudat"],
                "affected_parts": ["fruits", "cabosses", "branches"],
                "conditions": ["humidité élevée", "température 25-30°C", "blessures"],
                "severity": "élevée",
                "treatments": ["fongicides cupriques", "taille sanitaire", "amélioration drainage"]
            },
            {
                "name": "Mirides",
                "agent": "Sahlbergella singularis",
                "symptoms": ["taches noires", "dessèchement branches", "écoulement sève", "chancres"],
                "affected_parts": ["branches", "tronc", "rameaux"],
                "conditions": ["saison sèche", "stress hydrique", "mauvais entretien"],
                "severity": "très élevée",
                "treatments": ["insecticides", "taille parties atteintes", "amélioration ombrage"]
            }
        ]
    }
    
    # Calcul des scores de correspondance
    crop_diseases = disease_database.get(crop, [])
    disease_scores = []
    
    for disease in crop_diseases:
        score = 0
        total_criteria = 0
        
        # Score basé sur les symptômes
        if symptoms:
            symptom_matches = sum(1 for symptom in symptoms 
                                if any(s in symptom.lower() for s in disease["symptoms"]))
            score += (symptom_matches / len(disease["symptoms"])) * 40
            total_criteria += 40
        
        # Calcul du pourcentage de probabilité
        probability = (score / total_criteria * 100) if total_criteria > 0 else 0
        
        disease_scores.append({
            "disease": disease["name"],
            "agent": disease["agent"],
            "probability": probability,
            "severity": disease["severity"],
            "treatments": disease["treatments"],
            "matching_symptoms": [s for s in symptoms if any(ds in s.lower() for ds in disease["symptoms"])]
        })
    
    # Tri par probabilité décroissante
    disease_scores.sort(key=lambda x: x["probability"], reverse=True)
    
    return {
        "crop": crop,
        "symptoms": symptoms,
        "diagnostic_results": disease_scores,
        "most_likely_diagnosis": disease_scores[0] if disease_scores else None,
        "confidence_level": disease_scores[0]["probability"] if disease_scores else 0
    }
\end{lstlisting}

Les \textbf{outils agricoles spécialisés} démontrent la puissance d'ADK pour créer des fonctionnalités complexes accessibles via le langage naturel. Chaque outil encapsule une expertise spécifique tout en restant flexible dans son utilisation. Les outils analysent automatiquement les paramètres fournis par l'agent pour fournir des résultats pertinents.

La \textbf{conception des outils} suit des principes importants pour maximiser leur utilité. Les signatures de fonction claires avec typage permettent à l'agent de comprendre quand et comment utiliser l'outil. Les docstrings détaillées fournissent le contexte nécessaire pour une utilisation appropriée. Les paramètres optionnels offrent de la flexibilité tout en maintenant la simplicité pour les cas d'usage basiques.

\subsection{Passage de contexte entre agents}

Le passage efficace du contexte entre agents est crucial pour maintenir la cohérence des interactions et permettre une collaboration sophistiquée dans la résolution de problèmes complexes.

 
\begin{lstlisting}[language=Python, caption=Gestion du contexte partagé entre agents]
# agriculture_cameroun/tools.py
async def call_crops_agent(
    question: str,
    tool_context: ToolContext,
    crop: Optional[str] = None,
    region: Optional[str] = None,
):
    """Appelle l'agent cultures pour les questions de plantation et récolte.
    
    Args:
        question: Question sur les cultures
        crop: Culture spécifique (optionnel)
        region: Région spécifique (optionnel)
        tool_context: Contexte de l'outil
    
    Returns:
        Réponse de l'agent cultures
    """
    if region is None:
        region = tool_context.state["agriculture_settings"]["default_region"]
    
    agent_tool = AgentTool(agent=crops_agent)
    
    crops_input = {
        "request": question,
        "crop": crop,
        "region": region
    }
    
    response = await agent_tool.run_async(
        args=crops_input,
        tool_context=tool_context
    )
    
    tool_context.state["crops_response"] = response
    return response

async def call_health_agent(
    question: str,
    tool_context: ToolContext,
    symptoms: Optional[str] = None,
    crop: Optional[str] = None,
):
    """Appelle l'agent santé des plantes pour diagnostics et traitements.
    
    Args:
        question: Question sur la santé des plantes
        symptoms: Description des symptômes (optionnel)
        crop: Culture affectée (optionnel)
        tool_context: Contexte de l'outil
    
    Returns:
        Réponse de l'agent santé
    """
    agent_tool = AgentTool(agent=health_agent)
    
    health_input = {
        "request": question,
        "symptoms": symptoms,
        "crop": crop
    }
    
    response = await agent_tool.run_async(
        args=health_input,
        tool_context=tool_context
    )
    
    tool_context.state["health_response"] = response
    return response
\end{lstlisting}

Le \textbf{système de contexte partagé} permet aux agents de maintenir une compréhension cohérente de la conversation et des besoins de l'utilisateur. Le \texttt{tool\_context.state} centralise toutes les informations pertinentes, de l'historique conversationnel aux données partagées entre agents. Cette approche centralisée facilite la coordination tout en permettant à chaque agent de maintenir sa spécialisation.

L'\textbf{enrichissement contextuel} adapte dynamiquement le contexte selon les besoins spécifiques de chaque agent. Les réponses des agents sont stockées dans le contexte partagé, permettant aux appels suivants d'exploiter les informations précédemment obtenues.

\section{Implémentation de l'Agent Principal}

\subsection{Structure du fichier agent.py}

L'agent principal constitue le cœur du système Agriculture Cameroun, orchestrant l'ensemble des interactions et assurant la cohérence des réponses. Sa structure reflète cette responsabilité centrale tout en maintenant la modularité nécessaire pour l'évolution du système.

 
\begin{lstlisting}[language=Python, caption=Structure complète de l'agent principal]
# agriculture_cameroun/agent.py
import os
from datetime import date
from google.genai import types
from google.adk.agents import Agent
from google.adk.agents.callback_context import CallbackContext
from google.adk.tools import load_artifacts

from .sub_agents import (
    weather_agent,
    crops_agent,
    health_agent,
    economic_agent,
    resources_agent
)
from .prompts import return_instructions_root
from .tools import call_weather_agent, call_crops_agent, call_health_agent, call_economic_agent, call_resources_agent
from .utils.data import REGIONS, CROPS, SEASONS

date_today = date.today()

def setup_before_agent_call(callback_context: CallbackContext):
    """Configuration avant l'appel de l'agent."""
    
    # Configuration de l'état de session
    if "agriculture_settings" not in callback_context.state:
        agriculture_settings = {
            "regions": REGIONS,
            "crops": CROPS,
            "seasons": SEASONS,
            "current_date": date_today,
            "default_region": os.getenv("DEFAULT_REGION", "Centre"),
            "language": os.getenv("DEFAULT_LANGUAGE", "fr")
        }
        callback_context.state["agriculture_settings"] = agriculture_settings
    
    # Mise à jour des instructions avec le contexte
    context = callback_context.state["agriculture_settings"]
    callback_context._invocation_context.agent.instruction = (
        return_instructions_root()
        + f"""
        
        Contexte actuel:
        - Date: {context['current_date']}
        - Région par défaut: {context['default_region']}
        - Cultures principales: {', '.join(context['crops'].keys())}
        - Régions disponibles: {', '.join(context['regions'].keys())}
        """
    )

root_agent = Agent(
    model=os.getenv("ROOT_AGENT_MODEL"),
    name="agriculture_multiagent",
    instruction=return_instructions_root(),
    global_instruction=(
        f"""
        Tu es un Système Multi-Agents pour l'Agriculture Camerounaise.
        Date actuelle: {date_today}
        Langue: Français
        """
    ),
    sub_agents=[
        weather_agent,
        crops_agent,
        health_agent,
        economic_agent,
        resources_agent
    ],
    tools=[
        call_weather_agent,
        call_crops_agent,
        call_health_agent,
        call_economic_agent,
        call_resources_agent,
        load_artifacts,
    ],
    before_agent_callback=setup_before_agent_call,
    generate_content_config=types.GenerateContentConfig(temperature=0.7),
)
\end{lstlisting}

\subsection{Configuration et initialisation}

La configuration de l'agent principal suit une approche modulaire permettant une personnalisation facile selon les besoins spécifiques. Le fichier \texttt{config.py} centralise toutes les constantes et paramètres configurables du système.

 
\begin{lstlisting}[language=Python, caption=Configuration du système]
# agriculture_cameroun/config.py
from enum import Enum
from pydantic import BaseModel, Field

class RegionType(str, Enum):
    """Types de régions du Cameroun."""
    CENTRE = "Centre"
    LITTORAL = "Littoral"
    OUEST = "Ouest"
    SUD = "Sud"
    EST = "Est"
    NORD = "Nord"
    ADAMAOUA = "Adamaoua"
    EXTREME_NORD = "Extrême-Nord"
    NORD_OUEST = "Nord-Ouest"
    SUD_OUEST = "Sud-Ouest"

class CropType(str, Enum):
    """Types principaux de cultures."""
    CACAO = "cacao"
    CAFE = "café"
    MANIOC = "manioc"
    MAIS = "maïs"
    PLANTAIN = "plantain"
    ARACHIDE = "arachide"

class AgricultureConfig(BaseModel):
    """Configuration principale du système agricole."""
    
    default_region: RegionType = RegionType.CENTRE
    default_language: str = "fr"
    currency: str = "FCFA"
    
    # Modèles d'IA
    root_agent_model: str = Field(default="gemini-2.0-flash-001")
    weather_agent_model: str = Field(default="gemini-2.0-flash-001")
    crops_agent_model: str = Field(default="gemini-2.0-flash-001")
    health_agent_model: str = Field(default="gemini-2.0-flash-001")
    economic_agent_model: str = Field(default="gemini-2.0-flash-001")
    resources_agent_model: str = Field(default="gemini-2.0-flash-001")
\end{lstlisting}

L'initialisation du système suit une séquence précise garantissant que tous les composants sont correctement configurés avant le démarrage. Cette approche défensive permet de détecter les problèmes de configuration tôt et de fournir des messages d'erreur explicites.

\subsection{Routage vers les sous-agents}

Le mécanisme de routage constitue l'intelligence centrale du coordinateur, déterminant quels agents consulter pour chaque requête. Cette décision s'appuie sur l'analyse sémantique de la requête par Gemini, permettant une compréhension nuancée des besoins de l'utilisateur.

 
\begin{lstlisting}[language=Python, caption=Routage intelligent vers les agents]
# agriculture_cameroun/tools.py - Pattern de routage
async def call_economic_agent(
    question: str,
    tool_context: ToolContext,
    crop: Optional[str] = None,
    quantity: Optional[float] = None,
):
    """Appelle l'agent économique pour analyses de marché et rentabilité.
    
    Args:
        question: Question économique
        crop: Culture concernée (optionnel)
        quantity: Quantité en kg (optionnel)
        tool_context: Contexte de l'outil
    
    Returns:
        Réponse de l'agent économique
    """
    agent_tool = AgentTool(agent=economic_agent)
    
    economic_input = {
        "request": question,
        "crop": crop,
        "quantity": quantity
    }
    
    response = await agent_tool.run_async(
        args=economic_input,
        tool_context=tool_context
    )
    
    tool_context.state["economic_response"] = response
    return response
\end{lstlisting}

Le routage dans ADK se fait naturellement grâce à l'analyse sémantique du LLM qui comprend le contenu de la requête et sélectionne automatiquement les outils appropriés. Cette approche élimine le besoin de règles de routage explicites comme dans les systèmes traditionnels.

\section{Implémentation des Agents Spécialisés}

\subsection{Agent Météorologique}

L'Agent Météorologique fournit des informations climatiques essentielles pour la prise de décision agricole. Son implémentation combine accès aux données météo, analyse contextuelle et recommandations agricoles spécifiques.

 
\begin{lstlisting}[language=Python, caption=Outils météorologiques]
# sub_agents/weather/tools.py
def get_weather_forecast(
    region: str,
    tool_context: ToolContext,
    days: int = 7,
) -> Dict[str, Any]:
    """Obtient les prévisions météo pour une région.
    
    Args:
        region: Nom de la région camerounaise
        days: Nombre de jours de prévision (max 14)
        tool_context: Contexte de l'outil
        
    Returns:
        Prévisions météorologiques détaillées
    """
    # Simulation de données météo réalistes pour le Cameroun
    base_temp = {
        "Nord": {"min": 22, "max": 38},
        "Centre": {"min": 19, "max": 28},
        "Littoral": {"min": 23, "max": 31},
        "Ouest": {"min": 15, "max": 25},
        "Sud": {"min": 22, "max": 29}
    }
    
    # Obtenir les températures de base pour la région
    temps = base_temp.get(region, base_temp["Centre"])
    
    # Générer les prévisions
    forecast = []
    for i in range(min(days, 14)):
        date = datetime.now() + timedelta(days=i)
        
        # Variations aléatoires réalistes
        temp_min = temps["min"] + random.randint(-2, 2)
        temp_max = temps["max"] + random.randint(-2, 2)
        
        daily_forecast = {
            "date": date.strftime("%Y-%m-%d"),
            "temperature_min": temp_min,
            "temperature_max": temp_max,
            "humidity": random.randint(60, 85),
            "rain_probability": random.randint(10, 80),
            "wind_speed": random.randint(5, 20),
            "conditions": "Pluvieux" if random.randint(0, 100) > 50 else "Partiellement nuageux"
        }
        
        forecast.append(daily_forecast)
    
    return {
        "region": region,
        "forecast": forecast,
        "summary": f"Prévisions météo pour {region} sur {days} jours"
    }
\end{lstlisting}

L'agent météorologique utilise plusieurs outils spécialisés pour collecter et analyser les données météorologiques. Le système intègre des sources de données simulées qui peuvent être facilement remplacées par de vraies APIs météorologiques.

\subsection{Agent Cultures}

L'Agent Cultures apporte l'expertise agronomique au système, conseillant sur tous les aspects de la production végétale adaptée au contexte camerounais.

 
\begin{lstlisting}[language=Python, caption=Données agricoles camerounaises]
# agriculture_cameroun/utils/data.py
from agriculture_cameroun.config import RegionType, CropType

# Régions du Cameroun avec leurs caractéristiques
REGIONS = {
    RegionType.CENTRE: {
        "name": "Centre",
        "climate": "Équatorial de transition",
        "rainfall_mm": "1000-1600",
        "temperature_range": "22-28°C",
        "main_crops": ["manioc", "maïs", "plantain", "arachide"],
        "soil_types": ["argileux", "lateritique"],
        "agricultural_zones": ["Yaoundé", "Mbalmayo", "Obala"]
    },
    RegionType.LITTORAL: {
        "name": "Littoral",
        "climate": "Équatorial humide",
        "rainfall_mm": "1500-4000",
        "temperature_range": "24-30°C",
        "main_crops": ["cacao", "palmier_à_huile", "plantain", "manioc"],
        "soil_types": ["argileux", "sableux"],
        "agricultural_zones": ["Douala", "Edéa", "Nkongsamba"]
    }
}

# Prix moyens du marché (FCFA/kg)
MARKET_PRICES = {
    CropType.CACAO: {"min": 1000, "max": 1500, "average": 1200},
    CropType.CAFE: {"min": 1500, "max": 2500, "average": 2000},
    CropType.MANIOC: {"min": 150, "max": 300, "average": 200},
    CropType.MAIS: {"min": 200, "max": 400, "average": 300},
    CropType.PLANTAIN: {"min": 100, "max": 200, "average": 150},
    CropType.ARACHIDE: {"min": 600, "max": 1000, "average": 800}
}
\end{lstlisting}

La base de connaissances est structurée de manière hiérarchique, permettant à l'agent de naviguer efficacement des concepts généraux vers des recommandations spécifiques. Le fichier \texttt{data.py} contient des structures de données riches encodant l'expertise agricole camerounaise.

\subsection{Agent Santé des Plantes}

L'Agent Santé des Plantes agit comme phytopathologiste virtuel, diagnostiquant les problèmes et proposant des solutions intégrées.

 
\begin{lstlisting}[language=Python, caption=Base de données phytosanitaire]
# sub_agents/health/tools.py - Base de données des maladies
disease_database = {
    "cacao": [
        {
            "name": "Pourriture brune",
            "agent": "Phytophthora palmivora",
            "symptoms": ["taches brunes", "pourriture fruits", "brunissement cabosses"],
            "affected_parts": ["fruits", "cabosses", "branches"],
            "conditions": ["humidité élevée", "température 25-30°C"],
            "severity": "élevée",
            "treatments": ["fongicides cupriques", "taille sanitaire", "amélioration drainage"]
        },
        {
            "name": "Mirides",
            "agent": "Sahlbergella singularis",
            "symptoms": ["taches noires", "dessèchement branches", "écoulement sève"],
            "affected_parts": ["branches", "tronc", "rameaux"],
            "conditions": ["saison sèche", "stress hydrique"],
            "severity": "très élevée",
            "treatments": ["insecticides", "taille parties atteintes", "amélioration ombrage"]
        }
    ],
    "maïs": [
        {
            "name": "Striure du maïs",
            "agent": "Maize streak virus",
            "symptoms": ["striures jaunes", "nanisme", "déformation"],
            "affected_parts": ["feuilles", "plant entier"],
            "conditions": ["cicadelles vectrices"],
            "severity": "élevée",
            "treatments": ["variétés résistantes", "lutte contre cicadelles"]
        }
    ]
}
\end{lstlisting}

L'agent intègre une base de données complète des maladies et parasites communs au Cameroun. Cette base peut être facilement étendue avec de nouvelles maladies ou parasites selon les observations terrain.

\subsection{Agent Économique}

L'Agent Économique fournit l'intelligence commerciale nécessaire pour transformer l'agriculture de subsistance en entreprise rentable.

 
\begin{lstlisting}[language=Python, caption=Analyse économique]
# sub_agents/economic/tools.py
def analyze_profitability(
    crop: str,
    area_hectares: float,
    tool_context: ToolContext,
    production_system: str = "traditionnel",
) -> Dict[str, Any]:
    """Analyse la rentabilité d'une culture.
    
    Args:
        crop: Type de culture
        area_hectares: Superficie en hectares
        production_system: Système de production
        tool_context: Contexte de l'outil
        
    Returns:
        Analyse de rentabilité détaillée
    """
    # Coûts de base par hectare (FCFA)
    base_costs = {
        "semences": 25000,
        "engrais": 45000,
        "pesticides": 20000,
        "main_oeuvre": 80000,
        "transport": 15000,
        "divers": 10000
    }
    
    total_cost_per_ha = sum(base_costs.values())
    total_cost = total_cost_per_ha * area_hectares
    
    # Prix de vente estimé
    market_price = MARKET_PRICES.get(CropType(crop), {"average": 300})["average"]
    
    return {
        "crop": crop,
        "area_hectares": area_hectares,
        "costs": {
            "breakdown": base_costs,
            "total": total_cost
        },
        "market_price": market_price,
        "analysis": f"Analyse de rentabilité pour {crop} sur {area_hectares} ha"
    }
\end{lstlisting}

L'agent effectue des analyses financières complètes adaptées au contexte des petits agriculteurs camerounais. Les calculs prennent en compte les spécificités locales comme le travail familial et les systèmes d'entraide.

\subsection{Agent Ressources}

L'Agent Ressources optimise l'utilisation des ressources naturelles et des intrants, promouvant une agriculture durable et efficiente.

 
\begin{lstlisting}[language=Python, caption=Gestion des ressources]
# sub_agents/resources/tools.py
def analyze_soil_requirements(
    crop: str,
    region: str,
    tool_context: ToolContext,
    soil_type: Optional[str] = None,
    current_ph: Optional[float] = None,
) -> Dict[str, Any]:
    """Analyse les exigences du sol pour une culture donnée.
    
    Args:
        crop: Type de culture
        region: Région de culture
        soil_type: Type de sol (optionnel)
        current_ph: pH actuel du sol (optionnel)
        tool_context: Contexte de l'outil
        
    Returns:
        Analyse complète des exigences du sol
    """
    # Exigences par culture
    crop_requirements = {
        "cacao": {
            "ph_optimal": {"min": 6.0, "max": 7.0},
            "depth_cm": 150,
            "drainage": "bien drainé",
            "organic_matter_percent": {"min": 3.0, "optimal": 5.0}
        },
        "maïs": {
            "ph_optimal": {"min": 5.8, "max": 7.0},
            "depth_cm": 80,
            "drainage": "bien drainé à modéré",
            "organic_matter_percent": {"min": 2.0, "optimal": 4.0}
        }
    }
    
    requirements = crop_requirements.get(crop, crop_requirements["maïs"])
    
    return {
        "crop": crop,
        "region": region,
        "requirements": requirements,
        "soil_analysis": f"Analyse des exigences du sol pour {crop} en région {region}"
    }
\end{lstlisting}

L'agent génère des plans d'optimisation personnalisés considérant les contraintes et opportunités spécifiques de chaque exploitation. Cette approche modulaire avec des agents spécialisés permet au système Agriculture Cameroun d'offrir une expertise complète tout en maintenant la flexibilité nécessaire pour s'adapter aux besoins variés des agriculteurs camerounais.