
\begin{description}
    \item[Agent] Entité autonome capable de percevoir son environnement et d'agir de manière indépendante pour atteindre ses objectifs.
    
    \item[Autonomie] Capacité d'un agent à prendre des décisions et à agir sans intervention externe directe.
    
    \item[Comportement (Behavior)] Ensemble d'actions et de réactions d'un agent face aux stimuli de son environnement.
    
    \item[Communication inter-agents] Mécanisme permettant aux agents d'échanger des informations et de coordonner leurs actions.
    
    \item[Coordination] Processus par lequel les agents synchronisent leurs actions pour atteindre un objectif commun.
    
    \item[Émergence] Phénomène par lequel des propriétés complexes apparaissent au niveau système à partir d'interactions simples entre agents.
    
    \item[Environnement] Contexte dans lequel évoluent les agents, incluant les ressources et les contraintes.
    
    \item[LLM (Large Language Model)] Modèle d'intelligence artificielle capable de comprendre et générer du langage naturel.
    
    \item[Multi-agent] Système composé de plusieurs agents interagissant dans un environnement partagé.
    
    \item[Ontologie] Représentation formelle des connaissances d'un domaine spécifique.
    
    \item[Performative] Type d'acte de communication dans le langage ACL (ex: INFORM, REQUEST, PROPOSE).
    
    \item[Pro-activité] Capacité d'un agent à prendre des initiatives et à anticiper les besoins.
    
    \item[Réactivité] Capacité d'un agent à répondre rapidement aux changements de son environnement.
    
    \item[Socialité] Capacité d'un agent à interagir et collaborer avec d'autres agents.
    
    \item[Système expert] Système d'intelligence artificielle simulant le raisonnement d'un expert humain dans un domaine spécifique.
\end{description}