\chapter{INTÉGRATION ET DÉPLOIEMENT}

\section{Interface Utilisateur}

\subsection{Interface web avec ADK}

Google ADK révolutionne la création d'interfaces utilisateur pour les systèmes multi-agents en fournissant une interface web moderne et réactive out-of-the-box. Cette interface, basée sur les dernières technologies web, offre une expérience utilisateur fluide et intuitive parfaitement adaptée aux besoins des agriculteurs camerounais.

\begin{figure}[H]
\centering
\framebox[0.9\textwidth]{
\parbox{0.85\textwidth}{
\centering
\textbf{Interface Web Agriculture Cameroun}\\[10pt]
\textbf{Composants de l'interface ADK :}\\[5pt]
- Zone de chat interactive en bas\\
- Affichage des réponses avec formatage Markdown\\
- Indicateur d'agent actif en temps réel\\
- Support des suggestions de questions\\
- Upload de fichiers pour diagnostic\\[10pt]
URL d'accès : \texttt{http://localhost:8080}\\
Commande : \texttt{poetry run adk serve . --port 8080}
}
}
\caption{Architecture de l'interface web ADK}
\end{figure}

L'interface web ADK est automatiquement générée lors du lancement du système avec la commande \texttt{adk serve}. Cette approche zero-configuration permet de démarrer immédiatement sans configuration complexe d'interface.

\begin{lstlisting}[language=Python, caption=Configuration de l'agent principal pour l'interface]
# Extrait de agriculture_cameroun/agent.py
from google.adk.agents import Agent
from google.genai import types

# Configuration de l'agent principal
root_agent = Agent(
    model=os.getenv("ROOT_AGENT_MODEL", "gemini-2.0-flash-001"),
    name="agriculture_multiagent",
    instruction=return_instructions_root(),
    global_instruction=(
        f"""
        Tu es un Système Multi-Agents pour l'Agriculture Camerounaise.
        Date actuelle: {date_today}
        Langue: Français
        """
    ),
    sub_agents=[
        weather_agent,
        crops_agent,
        health_agent,
        economic_agent,
        resources_agent
    ],
    tools=[
        call_weather_agent,
        call_crops_agent,
        call_health_agent,
        call_economic_agent,
        call_resources_agent,
        load_artifacts,
    ],
    before_agent_callback=setup_before_agent_call,
    generate_content_config=types.GenerateContentConfig(temperature=0.7),
)
\end{lstlisting}

\subsection{API REST pour intégrations externes}

Au-delà de l'interface web, ADK expose automatiquement une API REST permettant l'intégration du système Agriculture Cameroun dans d'autres applications et services.

\begin{lstlisting}[language=Python, caption=Utilisation de l'API REST ADK]
# Exemples d'utilisation de l'API REST générée par ADK

# Point d'entrée principal pour les requêtes
# POST http://localhost:8080/v1/messages

# Format de requête JSON
{
    "messages": [{
        "role": "user",
        "content": "Quand planter le maïs dans la région Centre ?"
    }],
    "model": "agriculture_multiagent",
    "stream": false
}

# Headers requis
Content-Type: application/json
Accept: application/json

# Exemple avec curl
curl -X POST http://localhost:8080/v1/messages \
  -H "Content-Type: application/json" \
  -d '{
    "messages": [{
        "role": "user", 
        "content": "Prix actuel du cacao au marché"
    }]
  }'
\end{lstlisting}

\subsection{Exemples d'utilisation}

Pour illustrer concrètement l'utilisation du système, examinons le script de démonstration CLI fourni dans le projet.

\begin{lstlisting}[language=Python, caption=Interface CLI de démonstration (extrait de examples/demo\_cli.py)]
def interactive_mode():
    """Mode interactif pour tester les fonctionnalités."""
    print("🌱 AGRICULTURE CAMEROUN - MODE INTERACTIF")
    print("=========================================")
    print("Tapez 'aide' pour voir les commandes disponibles")
    print("Tapez 'quit' pour quitter\n")
    
    while True:
        try:
            user_input = input("Votre question: ").strip()
            
            if user_input.lower() in ['quit', 'exit', 'quitter']:
                print("Au revoir !")
                break
            elif user_input.lower() in ['aide', 'help']:
                print_help()
            elif not user_input:
                continue
            else:
                response = process_query(user_input)
                print(f"{response}\n")
                
        except KeyboardInterrupt:
            print("\nAu revoir ! ")
            break

def process_query(query: str) -> str:
    """Traite une requête utilisateur et retourne une réponse simulée."""
    query_lower = query.lower()
    
    # Détection du type de question
    if any(word in query_lower for word in ['météo', 'temps', 'pluie']):
        region = extract_region(query) or "Cameroun"
        return simulate_weather_query(region, query)
    
    elif any(word in query_lower for word in ['planter', 'culture', 'variété']):
        crop = extract_crop(query) or "culture"
        region = extract_region(query) or "Cameroun"
        return simulate_crop_query(crop, region)
\end{lstlisting}

\section{Tests et Validation}

\subsection{Tests unitaires des agents}

La stratégie de test du système Agriculture Cameroun assure la fiabilité et la précision des conseils fournis aux agriculteurs. Les tests sont organisés dans le répertoire \texttt{tests/}.

\begin{lstlisting}[language=Python, caption=Tests unitaires réels du projet (tests/test\_agents.py)]
import pytest
import os
from unittest.mock import Mock, patch, AsyncMock

# Configuration pour les tests
os.environ["GEMINI_API_KEY"] = "test-api-key"

from agriculture_cameroun.agent import root_agent
from agriculture_cameroun.sub_agents.weather.agent import weather_agent
from agriculture_cameroun.sub_agents.crops.agent import crops_agent

class TestAgentConfiguration:
    """Tests de configuration des agents."""
    
    def test_root_agent_initialization(self):
        """Test l'initialisation de l'agent principal."""
        assert root_agent.name == "agriculture_multiagent"
        assert root_agent.model is not None
        assert len(root_agent.sub_agents) == 5
        assert len(root_agent.tools) >= 6
    
    def test_sub_agents_initialization(self):
        """Test l'initialisation des sous-agents."""
        agents = [
            (weather_agent, "weather_agent"),
            (crops_agent, "crops_agent"),
        ]
        
        for agent, expected_name in agents:
            assert agent.name == expected_name
            assert agent.model is not None
            assert len(agent.tools) > 0

class TestWeatherAgent:
    """Tests pour l'agent météorologique."""
    
    @pytest.fixture
    def mock_weather_context(self):
        """Mock du contexte pour les outils météo."""
        context = Mock()
        context.state = {"agriculture_settings": {"default_region": "Centre"}}
        return context
    
    @patch('agriculture_cameroun.sub_agents.weather.tools.model.
    generate_content')
    def test_weather_forecast_tool(self, mock_generate, mock_weather_context):
        """Test l'outil de prévisions météo."""
        from agriculture_cameroun.sub_agents.weather.tools import get_weather_forecast
        
        mock_response = Mock()
        mock_response.text = "Prévisions météo favorables"
        mock_generate.return_value = mock_response
        
        result = get_weather_forecast(
            region="Centre",
            days=7,
            tool_context=mock_weather_context
        )
        
        assert "region" in result
        assert "forecast" in result
        assert result["region"] == "Centre"
\end{lstlisting}

\subsection{Tests d'intégration}

Les tests d'intégration vérifient la coordination entre les différents agents du système.

\begin{lstlisting}[language=Python, caption=Tests d'intégration des données (tests/test\_agents.py)]
class TestDataUtilities:
    """Tests pour les utilitaires de données."""
    
    def test_crop_info_retrieval(self):
        """Test la récupération d'informations sur les cultures."""
        from agriculture_cameroun.utils.data import get_crop_info
        from agriculture_cameroun.config import CropType
        
        cacao_info = get_crop_info(CropType.CACAO)
        assert cacao_info is not None
        assert cacao_info.name == CropType.CACAO
        assert cacao_info.growth_cycle_days > 0
        assert len(cacao_info.suitable_regions) > 0
    
    def test_region_info_retrieval(self):
        """Test la récupération d'informations sur les régions."""
        from agriculture_cameroun.utils.data import get_region_info
        from agriculture_cameroun.config import RegionType
        
        centre_info = get_region_info(RegionType.CENTRE)
        assert centre_info is not None
        assert "name" in centre_info
        assert "climate" in centre_info
        assert "main_crops" in centre_info
\end{lstlisting}

\subsection{Validation des données agricoles}

Un aspect crucial du système est la validation de l'exactitude des données agricoles camerounaises.

\begin{lstlisting}[language=Python, caption=Validation des données agricoles (agriculture\_cameroun/utils/data.py)]
# Données validées pour l'agriculture camerounaise
CROPS = {
    CropType.CACAO: CropInfo(
        name=CropType.CACAO,
        scientific_name="Theobroma cacao",
        local_names=["Cacao", "Cocoa"],
        growth_cycle_days=365,
        optimal_temperature_min=21,
        optimal_temperature_max=32,
        water_requirements="high",
        soil_preferences=[SoilType.ARGILEUX, SoilType.HUMIFERE],
        suitable_regions=[RegionType.CENTRE, RegionType.SUD, 
                         RegionType.LITTORAL, RegionType.SUD_OUEST],
        planting_seasons=[SeasonType.SAISON_PLUIES],
        expected_yield_per_hectare=600
    ),
    # ... autres cultures
}

# Prix moyens du marché validés (FCFA/kg)
MARKET_PRICES = {
    CropType.CACAO: {"min": 1000, "max": 1500, "average": 1200},
    CropType.CAFE: {"min": 1500, "max": 2500, "average": 2000},
    CropType.MANIOC: {"min": 150, "max": 300, "average": 200},
    # ... autres prix
}
\end{lstlisting}

\section{Déploiement}

\subsection{Déploiement local}

Le projet fournit des scripts de déploiement automatisé pour différentes plateformes.

\begin{lstlisting}[language=bash, caption=Script de déploiement local (setup.sh)]
#!/bin/bash
# Script d'installation automatique pour Agriculture Cameroun

set -e

# Couleurs pour les messages
RED='\033[0;31m'
GREEN='\033[0;32m'
YELLOW='\033[1;33m'
BLUE='\033[0;34m'
NC='\033[0m' # No Color

# Vérifier l'OS
check_os() {
    print_header "🔍 Vérification du système d'exploitation..."
    
    if [[ "$OSTYPE" == "linux-gnu"* ]]; then
        OS="linux"
        print_success "Linux détecté"
    elif [[ "$OSTYPE" == "darwin"* ]]; then
        OS="macos"
        print_success "macOS détecté"
    else
        print_error "Système d'exploitation non supporté: $OSTYPE"
        exit 1
    fi
}

# Vérifier Python
check_python() {
    print_header "🐍 Vérification de Python..."
    
    if command -v python3.12 &> /dev/null; then
        PYTHON_CMD="python3.12"
        print_success "Python 3.12 trouvé"
    elif command -v python3 &> /dev/null; then
        # Vérifier la version
        PYTHON_VERSION=$(python3 --version | cut -d' ' -f2)
        print_success "Python $PYTHON_VERSION trouvé"
    fi
}

# Installation du projet
install_project() {
    print_header "📦 Installation du projet..."
    
    # Clone ou mise à jour
    if [ -d "agriculture-cameroun" ]; then
        cd agriculture-cameroun
        git pull origin main
    else
        git clone https://github.com/Nameless0l/agriculture-cameroun.git
        cd agriculture-cameroun
    fi
    
    # Installation des dépendances
    poetry install --no-interaction --verbose
}
\end{lstlisting}

\subsection{Containerisation avec Docker}

Le projet inclut une configuration Docker complète pour faciliter le déploiement.

\begin{lstlisting}[language=Dockerfile, caption=Dockerfile du projet]
# Use Python 3.11 slim image for smaller size
FROM python:3.11-slim

# Set environment variables
ENV PYTHONUNBUFFERED=1 \
    PYTHONDONTWRITEBYTECODE=1 \
    PIP_NO_CACHE_DIR=1

# Set work directory
WORKDIR /app

# Install system dependencies
RUN apt-get update && apt-get install -y \
    curl \
    build-essential \
    && rm -rf /var/lib/apt/lists/*

# Install Poetry
RUN pip install poetry

# Configure Poetry
ENV POETRY_VENV_IN_PROJECT=1 \
    POETRY_CACHE_DIR=/tmp/poetry_cache

# Copy Poetry configuration files
COPY pyproject.toml poetry.lock* ./

# Install dependencies
RUN poetry install --only=main --no-root && rm -rf $POETRY_CACHE_DIR

# Copy application code
COPY agriculture_cameroun/ ./agriculture_cameroun/
COPY .env.example ./.env

# Create non-root user
RUN adduser --disabled-password --gecos '' appuser && \
    chown -R appuser:appuser /app
USER appuser

# Expose port
EXPOSE 8000

# Default command
CMD ["poetry", "run", "adk", "serve", ".", "--host", "0.0.0.0", "--port", "8000"]
\end{lstlisting}

\subsection{Déploiement en production avec Docker Compose}

Le fichier \texttt{docker-compose.yml} fourni permet un déploiement complet avec tous les services nécessaires.

\begin{lstlisting}[language=yaml, caption=Configuration Docker Compose (docker-compose.yml)]
version: '3.8'

services:
  agriculture-cameroun:
    build: .
    ports:
      - "8000:8000"
    environment:
      - LOG_LEVEL=INFO
      - ENABLE_CACHING=true
      - DATABASE_URL=sqlite:///./data/agriculture.db
    env_file:
      - .env
    volumes:
      - ./data:/app/data
      - ./logs:/app/logs
    restart: unless-stopped
    depends_on:
      - redis
    networks:
      - agriculture-network

  redis:
    image: redis:7-alpine
    ports:
      - "6379:6379"
    volumes:
      - redis-data:/data
    restart: unless-stopped
    networks:
      - agriculture-network

  nginx:
    image: nginx:alpine
    ports:
      - "80:80"
      - "443:443"
    volumes:
      - ./nginx.conf:/etc/nginx/nginx.conf
      - ./ssl:/etc/nginx/ssl
    depends_on:
      - agriculture-cameroun
    restart: unless-stopped
    networks:
      - agriculture-network

volumes:
  redis-data:

networks:
  agriculture-network:
    driver: bridge
\end{lstlisting}

\begin{figure}[H]
\centering
\framebox[0.9\textwidth]{
\parbox{0.85\textwidth}{
\centering
\textbf{Architecture de déploiement production}\\[10pt]
\textbf{Composants déployés :}\\[5pt]
- Container principal Agriculture Cameroun (ADK)\\
- Cache Redis pour les performances\\
- Reverse proxy Nginx pour SSL/TLS\\
- Volumes persistants pour données et logs\\
- Réseau Docker isolé\\[10pt]
\textbf{Commandes de déploiement :}\\[5pt]
\texttt{docker-compose up -d} - Démarrer tous les services\\
\texttt{docker-compose ps} - Vérifier l'état\\
\texttt{docker-compose logs -f} - Suivre les logs\\
\texttt{docker-compose down} - Arrêter les services
}
}
\caption{Stack de déploiement Docker Compose}
\end{figure}

Cette architecture garantit que le système Agriculture Cameroun peut être déployé facilement tout en maintenant les performances et la fiabilité nécessaires pour servir les agriculteurs camerounais.