% Packages and document configurations.
\documentclass[en]{IPLeiriaThesis}
\usepackage{microtype}

% Document variables.
% First Author (Mandatory)
\firstauthor{Joe Smith}
\firstauthorid{2230455}

% Second Author (Optional)
% \secondauthor{Jane Smith}
% \secondauthorid{2230456}

% Third Author (Optional)
% \thirdauthor{July Smith}
% \thirdauthorid{2230457}

% Supervisor (Mandatory)
\supervisor{John Smith}
\supervisormail{joe.smith@ipleiria.pt}
\supervisortitle{Full Professor, Polytechnic of Leiria} % Please provide: [Current Title, Affiliation]

% Co-Supervisor (Optional)
\cosupervisor{Steve Smith}
\cosupervisormail{steve.smith@ipleiria.pt}
\cosupervisortitle{Associate Professor, Polytechnic of Leiria}

% Second Co-Supervisor (Optional)
\seccosupervisor{Shak Smith}
\seccosupervisormail{shak.smith@ipleiria.pt}
\seccosupervisortitle{Associate Researcher, Computer Science \& Communication Research Centre}

% Title (Mandatory)
\title{Improving Machine Learning Efficiency Against Noisy Data Sources}

% Subtitle (Mandatory)
\subtitle{Investigating Advanced Strategies to Mitigate Adverse Effects of Noisy Data}

% University (Mandatory)
\university{Polytechnic of Leiria}

% School (Mandatory)
\school{School of Management and Technology}

% Department (Mandatory)
\department{Department of Computer Engineering}

% Degree (Mandatory)
\degree{Master in Cybersecurity \& Digital Forensics}

% Course (Optional)
% \course{Offensive \& Defensive Cybersecurity}

% Thesis Theme (Mandatory)
\thesistheme{Dissertation/Project/Internship \\ %
\textcolor{blue}{(Erase the Non-Essential)}}

% Local & Date (Mandatory)
\date{Leiria, \DTMmonthname{\month} \number\year}

% Loading of the glossary and acronyms.
\makeglossaries
\loadglsentries{Matter/04-Glossary}
\loadglsentries[\acronymtype]{Matter/05-Acronyms}

\begin{document}

% Front matter.
% \ifthenelse{\equal{\getLanguage}{portuguese}}{%
    \pdfbookmark[0]{Capa}{capa} % Add entry to PDF bookmarks group.
    \pdfbookmark[1]{Frontispício}{frontispício} % Add entry to PDF.
}{%
    \pdfbookmark[0]{Front Matter}{frontmatter} % Add entry to PDF bookmarks group.
    \pdfbookmark[1]{Cover}{cover} % Add entry to PDF.
}

% Add background picture depending on the 'bwcover' variable.
\ifbwcover
    \newcommand\BackgroundPicCover{%
    \put(0,0){%
    \parbox[b][\paperheight]{\paperwidth}{%
    \vfill
    \centering
    \includegraphics[width=\paperwidth,height=\paperheight,keepaspectratio]{Figures/Theme/Front-Page-BG.pdf}%
    \vfill
    }}}
\else
    \newcommand\BackgroundPicCover{%
    \put(0,0){%
    \parbox[b][\paperheight]{\paperwidth}{%
    \vfill
    \centering
    \includegraphics[width=\paperwidth,height=\paperheight,keepaspectratio]{Figures/Theme/Cover-BG.pdf}%
    \vfill
    }}}
\fi

\AddToShipoutPictureBG*{\BackgroundPicCover}

\newgeometry{margin=1.98cm, top=2.15cm, bottom=1.47cm}
\begin{titlepage}
    \latofont % Switch to Lato font for the title page.
   
    \ifbwcover
        \color{frontpagedark} % Define the color to use within this page.
    \else
        \color{white} % Define the color to use within this page.
    \fi
    
    \vspace*{\baselineskip} % White space at the top of the page.

    \ifbwcover
        \begin{figure}
            \includegraphics[width=0.32\linewidth]{Figures/Theme/IPLeiria-Logo-B.pdf}
        \end{figure}
    \else
        \begin{figure}
            \includegraphics[width=0.32\linewidth]{Figures/Theme/IPLeiria-Logo-W.pdf}
        \end{figure}
    \fi

    \vspace{5.5\baselineskip}

    % Title
	\noindent
    \makebox[\textwidth][l]{%
        \parbox{\dimexpr\textwidth-2.5cm\relax}{
            \setstretch{1.03}
            \raggedright\bfseries\fontsize{20}{26}\selectfont\thetitle
        }
    }

    \vspace{0.8\baselineskip}

    % Subtitle
    \noindent
    \makebox[\textwidth][l]{%
        \parbox{\dimexpr\textwidth-7cm\relax}{
            \setstretch{1.03}
            \raggedright\fontsize{14}{19}\selectfont\subname
        }
    }

    \vspace{35pt}  

    % Author
    {\noindent\bfseries\fontsize{14}{19}\selectfont\firstauthorname}

    \ifdefined\secondauthorname
        \vspace{8pt}
        {\noindent\bfseries\fontsize{14}{19}\selectfont\secondauthorname}
	\fi

    \ifdefined\thirdauthorname
        \vspace{8pt}
        {\noindent\bfseries\fontsize{14}{19}\selectfont\thirdauthorname}
	\fi
 
	\vfill

    % School
	{\noindent\fontsize{10}{12}\selectfont\schoolname}
	
    % Department
	{\noindent\fontsize{10}{12}\selectfont\departmentname}

    % Degree
	{\noindent\fontsize{10}{12}\selectfont\degname}

    % Course
    \ifdefined\coursename
        {\noindent\fontsize{10}{12}\selectfont\coursename}
	\fi

    \vspace{125pt}

    % Local & Date
	{\noindent\fontsize{10}{12}\selectfont\thedate}

    \vspace{68pt}
\end{titlepage}
\restoregeometry\blankpage
% \ifthenelse{\equal{\getLanguage}{portuguese}}{%
    \pdfbookmark[1]{Primeira Página}{primeirapagina} % Add entry to PDF.
}{%
    \pdfbookmark[1]{Front Page}{frontpage} % Add entry to PDF.
}

% Add background picture.
\newcommand\BackgroundPicFrontPage{%
    \put(0,0){%
    \parbox[b][\paperheight]{\paperwidth}{%
    \vfill
    \centering
    \includegraphics[width=\paperwidth,height=\paperheight,keepaspectratio]{Figures/Theme/Front-Page-BG.pdf}%
    \vfill
}}}
\AddToShipoutPictureBG*{\BackgroundPicFrontPage}

\newgeometry{margin=1.98cm, top=2.15cm, bottom=1.47cm}
\begin{titlepage}
    \latofont % Switch to Lato font for the title page.
    \color{frontpagedark} % Define the color to use within this page.
    \vspace*{\baselineskip} % White space at the top of the page.

    \begin{figure}
        \includegraphics[width=0.32\linewidth]{Figures/Theme/IPLeiria-Logo-B.pdf}
    \end{figure}

    \vspace{3.5\baselineskip}

    % Title.
	\noindent
    \makebox[\textwidth][l]{%
        \parbox{\dimexpr\textwidth-2.5cm\relax}{
            \setstretch{1.03}
            \raggedright\bfseries\fontsize{20}{26}\selectfont\thetitle
        }
    }

    \vspace{0.8\baselineskip}

    % Subtitle.
    \noindent
    \makebox[\textwidth][l]{%
        %\fbox
    }

    \vspace{35pt}

    % Author.
    {\noindent\bfseries\fontsize{14}{19}\selectfont\firstauthorname}

    \ifdefined\secondauthorname
        \vspace{8pt}
        {\noindent\bfseries\fontsize{14}{19}\selectfont\secondauthorname}
	\fi

    \ifdefined\thirdauthorname
        \vspace{8pt}
        {\noindent\bfseries\fontsize{14}{19}\selectfont\thirdauthorname}
	\fi

    \vspace{70pt}    

    {
    \noindent
    \latofont
    \fontsize{10}{12}\selectfont
    \renewcommand{\arraystretch}{0.1}
    \hspace*{-2.5pt}\begin{tabular}{@{}r@{\hspace{5pt}}>{\raggedright\arraybackslash}m{6cm}@{}}
        \textbf{Supervisor:} & \supname \\ [-.7ex]
        & \setstretch{0.9}{\fontsize{8}{10}\selectfont\itshape \suptitle} \\ [2ex]
        
        \ifdefined\cosupname
            \textbf{Co-supervisor:} & \cosupname \\ [-.7ex]
            & \setstretch{0.9}{\fontsize{8}{10}\selectfont\itshape \cosuptitle} \\ [.5ex]
        \fi

        \ifdefined\seccosupname        
            & \seccosupname \\ [-.7ex]
            & \setstretch{0.9}{\fontsize{8}{10}\selectfont\itshape \seccosuptitle} \\
        \fi
    \end{tabular}
    }
    
    \vfill
	
    % School.
	{\noindent\fontsize{10}{12}\selectfont\schoolname}
	
    % Department.
	{\noindent\fontsize{10}{12}\selectfont\departmentname}

    % Degree.
	{\noindent\fontsize{10}{12}\selectfont\degname}

    % Course.
    \ifdefined\coursename
        {\noindent\fontsize{10}{12}\selectfont\coursename}
	\fi

    \vspace{45pt}

    % Thesis option.
	{\noindent\fontsize{10}{12}\itshape\selectfont\thesisthemeop}

    \vspace{45pt}

    % Local and date.
	{\noindent\fontsize{10}{12}\selectfont\thedate}

    \vspace{68pt}
\end{titlepage}
\restoregeometry\blankpage

% Roman numeration.
\pagenumbering{roman}

% Declaration of authorship.
% \thispagestyle{plain} % Page style without header and footer

\ifthenelse{\equal{\getLanguage}{portuguese}}{%
    \pdfbookmark[1]{Declaração}{declaração} % Add entry to PDF
    \chapter*{Declaração de Autoria} % Chapter* to appear without numeration
}{%
    \pdfbookmark[1]{Declaration}{declaration} % Add entry to PDF
    \chapter*{Declaration of Authorship} % Chapter* to appear without numeration
}

\ifthenelse{\equal{\getLanguage}{portuguese}}{%
    O abaixo assinado declara que o presente trabalho intitulado ``\thetitle'' é um trabalho original e que não foi anteriormente apresentado, na sua totalidade ou em parte, a nenhuma universidade ou instituição de ensino superior para a obtenção de qualquer grau, diploma ou outras qualificações.
    Declara-se igualmente que, tanto quanto é do seu conhecimento, este trabalho não contém material previamente publicado ou escrito por outra pessoa, exceto quando é feita a devida referência, reconhecimento e citação.%
}{%
    Has undersigned, hereby it his declared that this work entitled ``\thetitle'' is the original work and that it has not previously in its entirety or in part been submitted at any university or higher education institution for the award of any degree, diploma, or other qualifications. It is also hereby declared that to the best of the knowledge, this work contains no material previously published or written by another person, except where due reference, acknowledgement, and citation is made.%
}

\vspace{2.5\baselineskip}

{\noindent\textit{\thedate}}

\vspace{2\baselineskip}

\begin{flushright}
    \begin{tabular}{m{7cm}}
        \hrulefill \\
        \centering\firstauthorname \\ [8ex]
        
        \ifdefined\secondauthorname
            \hrulefill \\
            \centering\secondauthorname \\ [8ex]
        \fi

        \ifdefined\thirdauthorname
            \hrulefill \\
            \centering\thirdauthorname \\ [8ex]
        \fi
    \end{tabular}
\end{flushright}
\plainblankpage

% Acknowledgements.
% \thispagestyle{plain} % Page style without header and footer

\ifthenelse{\equal{\getLanguage}{portuguese}}{%
    \pdfbookmark[1]{Agradecimentos}{agradecimentos} % Add entry to PDF
    \chapter*{Agradecimentos} % Chapter* to appear without numeration
}{%
    \pdfbookmark[1]{Acknowledgements}{acknowledgements} % Add entry to PDF
    \chapter*{Acknowledgements} % Chapter* to appear without numeration
}

\blindtext
\plainblankpage

% Abstract.
% \include{Chapters/00-Abstract}

% List of contents, figures, and tables.
\tcbset{
    definition/.style={
        colback=blue!5,
        colframe=blue!75!black,
        fonttitle=\bfseries,
        title={Définition:}
    },
    theorem/.style={
        colback=green!5,
        colframe=green!75!black,
        fonttitle=\bfseries,
        title={Théorème:}
    },
    example/.style={
        colback=orange!5,
        colframe=orange!75!black,
        fonttitle=\bfseries,
        title={Exemple:}
    }
}

\lstdefinestyle{mystyle}{
    language=Java,
    basicstyle=\ttfamily\small,
    keywordstyle=\color{blue}\bfseries,
    commentstyle=\color{green!50!black},
    stringstyle=\color{red},
    numbers=left,
    numberstyle=\tiny\color{gray},
    stepnumber=1,
    numbersep=5pt,
    backgroundcolor=\color{gray!10},
    frame=single,
    breaklines=true,
    tabsize=4
}

% Appliquer le style à tous les listings
\lstset{style=mystyle}

% Mots-clés personnalisés pour vos structures
\lstset{morekeywords={STRUCTURE, FIN, LISTE_ALTERNÉE, ÉTAT, ACTION, LISTE_DE_PAIRES, CHAÎNE, PERCEPTION, ENVIRONNEMENT, N-UPLET, FONCTION, ENUM, DICTIONNAIRE}}

\tableofcontents
\listoffigures
\listoflistings
% \listoftables
\newpage

% Print glossary and acronyms.
% \printglossary\plainblankpage
% \printglossary[type=\acronymtype]\plainblankpage

% Arabic numeration.
\pagenumbering{arabic}

% Chapters.
\include{Chapters/abbreviations}
\chapter*{Introduction}

Dans ce tutoriel, nous allons explorer le développement de \textbf{systèmes multi-agents (SMA)} à travers la création d'un système d'assistance agricole pour le Cameroun. Nous utiliserons \textbf{Google Agent Development Kit (ADK)}, un framework moderne qui permet de créer des agents intelligents capables de collaborer pour résoudre des problèmes complexes. Ce document vous guidera pas à pas depuis les concepts fondamentaux jusqu'à l'implémentation complète d'un système fonctionnel.

\section*{Objectifs du tutoriel}

Ce tutoriel vise à vous fournir une compréhension approfondie des \textbf{systèmes multi-agents} et la maîtrise pratique de \textbf{Google ADK}. Vous apprendrez d'abord les \emph{concepts fondamentaux} qui sous-tendent les SMA, notamment les notions d'\textbf{agent}, d'\textbf{autonomie}, de \textbf{communication inter-agents}, de \textbf{protocoles d'interaction} et d'\textbf{ontologies}. Cette base théorique solide vous permettra de comprendre comment les agents peuvent collaborer efficacement pour résoudre des problèmes complexes.

Vous découvrirez ensuite \textbf{Google ADK}, un framework moderne qui révolutionne le développement d'agents intelligents grâce à son intégration native avec les \emph{modèles de langage}. Pour ceux ayant déjà une expérience avec \textbf{JADE}, nous établirons des parallèles et soulignerons les différences majeures entre ces deux approches, facilitant ainsi la transition vers ce nouveau paradigme.

L'objectif principal reste l'\emph{implémentation pratique} d'un système multi-agents complet pour l'agriculture camerounaise. Vous construirez progressivement \textbf{cinq agents spécialisés} (\emph{Météorologique}, \emph{Cultures}, \emph{Santé des Plantes}, \emph{Économique} et \emph{Ressources}), coordonnés par un \textbf{agent principal}. Cette approche pratique vous permettra de maîtriser les mécanismes de communication inter-agents, les protocoles d'interaction et les stratégies de coordination. Enfin, vous apprendrez à \textbf{déployer} et \textbf{tester} votre système dans un environnement réel.

\section*{Public cible et prérequis}

Ce tutoriel s'adresse principalement aux \textbf{étudiants en informatique} suivant un cours sur les systèmes multi-agents, mais également aux \textbf{développeurs} souhaitant découvrir cette technologie et aux \textbf{professionnels du secteur agricole} intéressés par les solutions intelligentes. La progression pédagogique a été conçue pour accompagner différents niveaux d'expertise, depuis les concepts de base jusqu'aux implémentations avancées.

Pour tirer pleinement profit de ce tutoriel, vous devez posséder une connaissance solide du langage \textbf{Python}, incluant la \emph{programmation orientée objet}, la gestion des \emph{modules} et \emph{packages}. Une compréhension des concepts de base en \textbf{intelligence artificielle}, notamment les notions d'agents intelligents et de systèmes distribués, facilitera grandement votre apprentissage. L'expérience avec un \textbf{environnement de développement intégré} comme \emph{VS Code} et la familiarité avec les outils en \emph{ligne de commande} sont également nécessaires. Des notions de base de \textbf{Git} vous permettront de cloner le dépôt du projet et de suivre les exemples de code.

Sur le plan matériel, assurez-vous de disposer d'un ordinateur sous \emph{Windows 10/11}, \emph{macOS 10.15+} ou \emph{Linux Ubuntu 20.04+}, avec \textbf{Python 3.12} ou une version supérieure installée. Un minimum de \textbf{8 GB de RAM} est requis, bien que 16 GB soient recommandés pour une expérience optimale. Prévoyez environ \textbf{2 GB d'espace disque} disponible et une \emph{connexion Internet stable} pour les téléchargements et l'accès aux API externes.

\section*{Vue d'ensemble du projet Agriculture Cameroun}

Le projet \textbf{Agriculture Cameroun} représente une réponse innovante aux défis complexes du secteur agricole camerounais. Dans un contexte où les agriculteurs font face à une \emph{variabilité climatique} croissante, des \emph{maladies des cultures} imprévisibles, des \emph{fluctuations des prix} du marché et une gestion souvent sub-optimale des \emph{ressources limitées}, l'accès à une information fiable et personnalisée devient crucial pour la prise de décision.

Les agriculteurs camerounais rencontrent quotidiennement des \textbf{obstacles majeurs} dans leur activité. L'accès aux \emph{informations météorologiques} fiables et localisées reste limité, rendant difficile la planification des activités agricoles. Le \emph{diagnostic des maladies} des plantes et le choix des \emph{traitements appropriés} constituent un défi constant, souvent aggravé par le manque d'expertise technique disponible localement. Les informations sur les \emph{prix du marché} et la \emph{rentabilité} des différentes cultures sont fragmentées et peu accessibles, compliquant les décisions économiques. La gestion des ressources précieuses comme l'\textbf{eau}, les \textbf{engrais} et les \textbf{semences} se fait souvent de manière empirique, sans optimisation réelle. Enfin, l'absence de \emph{conseils personnalisés} adaptés au contexte spécifique de chaque exploitation limite le potentiel de productivité.

Face à ces défis, notre système multi-agents propose une \textbf{plateforme intelligente} où cinq agents spécialisés collaborent harmonieusement. L'\textbf{Agent Météorologique} collecte et analyse les données climatiques pour fournir des prévisions localisées et des alertes pertinentes. L'\textbf{Agent Cultures} s'appuie sur une base de connaissances agronomiques pour conseiller sur les pratiques culturales optimales et les périodes de semis. L'\textbf{Agent Santé des Plantes} utilise des techniques de reconnaissance et d'analyse pour diagnostiquer les problèmes phytosanitaires et proposer des traitements adaptés. L'\textbf{Agent Économique} analyse les tendances du marché et aide à évaluer la rentabilité des différentes options culturales. L'\textbf{Agent Ressources} optimise l'utilisation des intrants agricoles en proposant des stratégies de gestion durable.

Ces agents ne fonctionnent pas en isolation mais sont orchestrés par un \textbf{agent coordinateur principal} qui joue un rôle crucial dans le système. Cet agent reçoit les requêtes des utilisateurs formulées en \emph{langage naturel}, analyse leur intention pour déterminer quels agents spécialisés solliciter, coordonne les interactions entre agents pour les requêtes complexes, et synthétise les différentes réponses pour fournir une information cohérente et directement actionnable par l'agriculteur.

L'utilisation de \textbf{Google ADK} comme framework de développement apporte une dimension moderne au projet. L'intégration native avec les \emph{modèles de langage Gemini} permet des interactions naturelles et intuitives avec les utilisateurs. La capacité d'analyser et de traiter des \emph{données complexes} provenant de sources multiples enrichit considérablement la qualité des recommandations. L'architecture flexible d'ADK facilite l'ajout de nouveaux agents ou l'extension des capacités existantes selon l'évolution des besoins.
\chapter{ CONCEPTS FONDAMENTAUX DES SYSTÈMES MULTI-AGENTS}

\section{Introduction aux Systèmes Multi-Agents (SMA)}

\subsection{Définition et caractéristiques d'un SMA}

Un \textbf{Système Multi-Agents (SMA)} représente une approche révolutionnaire en informatique qui s'inspire des organisations sociales pour résoudre des problèmes complexes. Contrairement aux systèmes traditionnels centralisés où un seul programme contrôle l'ensemble des opérations, un SMA est composé de plusieurs entités autonomes appelées \emph{agents} qui coexistent, interagissent et collaborent dans un environnement partagé pour atteindre des objectifs individuels ou collectifs.

Pour comprendre véritablement ce qu'est un SMA, imaginez une équipe de spécialistes travaillant sur un projet complexe. Chaque membre possède ses propres compétences, sa propre vision du problème et ses propres méthodes de travail. Ils communiquent entre eux, partagent des informations, négocient des solutions et coordonnent leurs actions pour atteindre un objectif commun. Un SMA fonctionne exactement de cette manière, mais avec des agents logiciels plutôt que des humains.

Les \textbf{caractéristiques fondamentales} d'un SMA incluent la \emph{distribution} du contrôle et des connaissances entre plusieurs agents, où aucun agent ne possède une vue complète du système ou ne peut contrôler entièrement son comportement. Cette distribution apporte une \emph{robustesse} remarquable au système, car la défaillance d'un agent n'entraîne pas nécessairement l'échec du système entier. Les autres agents peuvent compenser, réorganiser leurs interactions ou trouver des solutions alternatives.

La \emph{modularité} constitue une autre caractéristique essentielle des SMA. Chaque agent représente un module indépendant avec ses propres responsabilités, facilitant ainsi le développement, la maintenance et l'évolution du système. Cette modularité permet d'ajouter ou de retirer des agents selon les besoins, offrant une \emph{flexibilité} et une \emph{scalabilité} difficiles à atteindre avec des architectures monolithiques.

L'\emph{émergence} de comportements complexes à partir d'interactions simples entre agents représente l'un des aspects les plus fascinants des SMA. Des agents suivant des règles relativement simples peuvent, par leurs interactions, produire des comportements sophistiqués et des solutions innovantes que personne n'avait explicitement programmées. Cette propriété émergente rappelle les phénomènes observés dans la nature, comme l'organisation des colonies de fourmis ou les mouvements coordonnés des bancs de poissons.

\subsection{Notion d'agent : autonomie, réactivité, pro-activité, socialité}

Un \textbf{agent} dans le contexte des SMA est bien plus qu'un simple programme informatique. Il s'agit d'une entité computationnelle sophistiquée qui perçoit son environnement, prend des décisions et agit pour atteindre ses objectifs. Pour qu'une entité logicielle soit considérée comme un agent, elle doit posséder quatre propriétés fondamentales qui définissent son essence même.

L'\textbf{autonomie} constitue la première et peut-être la plus importante caractéristique d'un agent. Un agent autonome opère sans intervention directe humaine ou d'autres agents et possède un contrôle sur ses actions et son état interne. Cette autonomie ne signifie pas l'isolation totale, mais plutôt la capacité de prendre des décisions indépendantes basées sur ses connaissances, ses objectifs et sa perception de l'environnement. Par exemple, dans notre système agricole, l'Agent Météorologique décide de manière autonome quand collecter des données, comment les analyser et quand alerter les autres agents de conditions météorologiques critiques.

La \textbf{réactivité} permet à l'agent de percevoir son environnement et de répondre en temps opportun aux changements qui s'y produisent. Un agent réactif maintient une vigilance constante sur son environnement, détecte les modifications pertinentes et ajuste son comportement en conséquence. Cette réactivité est cruciale pour maintenir la pertinence et l'efficacité de l'agent dans un environnement dynamique. L'Agent Santé des Plantes, par exemple, doit réagir rapidement lorsqu'il détecte des symptômes de maladie dans les données qu'il analyse, déclenchant immédiatement un processus de diagnostic et de recommandation de traitement.

La \textbf{pro-activité} distingue les agents intelligents des simples programmes réactifs. Un agent pro-actif ne se contente pas de réagir aux événements ; il prend des initiatives, anticipe les besoins futurs et agit pour atteindre ses objectifs sans attendre des stimuli externes. Cette capacité d'initiative permet aux agents de planifier, d'optimiser leurs actions et de contribuer activement à la résolution de problèmes. L'Agent Économique illustre parfaitement cette propriété lorsqu'il analyse les tendances du marché pour anticiper les fluctuations de prix et conseiller proactivement les agriculteurs sur les meilleures périodes de vente.

La \textbf{socialité} reflète la capacité des agents à interagir avec d'autres agents (et éventuellement avec des humains) à travers un langage de communication commun. Cette dimension sociale permet la collaboration, la négociation, la coordination et le partage d'informations entre agents. Un agent social comprend les protocoles de communication, respecte les conventions d'interaction et peut s'engager dans des dialogues complexes pour résoudre des problèmes collectivement. Dans notre système, tous les agents communiquent entre eux pour fournir des recommandations cohérentes et complètes aux agriculteurs.

\subsection{Architecture des SMA : agents, environnement, interactions}

L'architecture d'un SMA repose sur trois composants fondamentaux interdépendants qui définissent la structure et le fonctionnement du système. Comprendre ces composants et leurs relations est essentiel pour concevoir et implémenter des SMA efficaces.

Les \textbf{agents} constituent le premier composant, représentant les entités actives du système. Chaque agent possède sa propre architecture interne qui peut varier considérablement selon sa complexité et ses responsabilités. Les architectures d'agents les plus courantes incluent les \emph{agents réactifs simples} qui répondent directement aux stimuli selon des règles prédéfinies, les \emph{agents délibératifs} qui maintiennent une représentation symbolique du monde et planifient leurs actions, et les \emph{agents hybrides} qui combinent réactivité et délibération pour allier efficacité et sophistication. Dans notre système agricole, l'Agent Coordinateur Principal adopte une architecture hybride, capable de réagir rapidement aux requêtes tout en planifiant la coordination des autres agents.

L'\textbf{environnement} représente le monde dans lequel les agents existent et opèrent. Il peut être \emph{physique} (comme un réseau de capteurs agricoles), \emph{virtuel} (comme une base de données), ou \emph{mixte}. L'environnement définit les conditions d'existence des agents, les ressources disponibles, les contraintes opérationnelles et les possibilités d'action. Dans notre projet, l'environnement comprend les données météorologiques, les informations sur les cultures, les prix du marché, et l'état des exploitations agricoles. Cet environnement est \emph{dynamique}, changeant continuellement avec les conditions météorologiques, les cycles agricoles et les fluctuations du marché.

Les \textbf{interactions} entre agents constituent le troisième pilier de l'architecture SMA. Ces interactions peuvent prendre diverses formes, de la simple communication d'informations à la négociation complexe, en passant par la coopération, la compétition ou la coordination. Les mécanismes d'interaction définissent comment les agents échangent des informations, synchronisent leurs actions, résolvent les conflits et atteignent des consensus. Dans notre système, les interactions sont principalement \emph{coopératives}, les agents partageant leurs connaissances spécialisées pour fournir des recommandations complètes aux agriculteurs.

L'architecture globale d'un SMA doit également considérer l'\emph{organisation} des agents, qui peut être \emph{hiérarchique} (avec des relations de subordination), \emph{hétérarchique} (sans hiérarchie fixe), ou \emph{hybride}. Notre système adopte une organisation hybride avec l'Agent Coordinateur Principal servant de point central de coordination sans pour autant exercer un contrôle hiérarchique strict sur les agents spécialisés.

\subsection{Domaines d'application des SMA}

Les systèmes multi-agents ont trouvé des applications dans une variété impressionnante de domaines, démontrant leur polyvalence et leur efficacité pour résoudre des problèmes complexes nécessitant distribution, autonomie et adaptation.

Dans le domaine de l'\textbf{agriculture intelligente}, qui est le focus de notre tutoriel, les SMA révolutionnent la gestion des exploitations agricoles. Au-delà de notre système d'assistance aux agriculteurs camerounais, les SMA sont utilisés pour l'optimisation de l'irrigation, la gestion des serres automatisées, la surveillance des cultures par drones, et la coordination des machines agricoles autonomes. Ces applications permettent une agriculture de précision, réduisant les coûts et l'impact environnemental tout en maximisant les rendements.

Le secteur des \textbf{transports et de la logistique} bénéficie grandement des SMA pour la gestion du trafic urbain, l'optimisation des chaînes d'approvisionnement et la coordination des véhicules autonomes. Les agents représentant des véhicules, des infrastructures routières et des centres de contrôle collaborent pour minimiser les embouteillages, optimiser les itinéraires et améliorer la sécurité routière. Dans les ports et aéroports, les SMA coordonnent les mouvements de marchandises, l'allocation des ressources et la planification des opérations.

Les \textbf{marchés financiers} utilisent intensivement les SMA pour le trading automatisé, l'analyse de risques et la détection de fraudes. Des agents spécialisés surveillent les marchés, analysent les tendances, exécutent des transactions et ajustent les portefeuilles en temps réel. La nature distribuée des SMA permet de traiter d'énormes volumes de données financières et de réagir aux changements du marché avec une rapidité impossible pour les traders humains.

Dans le domaine de la \textbf{santé}, les SMA assistent le diagnostic médical, la gestion hospitalière et le suivi des patients. Des agents représentant différents spécialistes médicaux peuvent collaborer pour établir des diagnostics complexes, tandis que d'autres agents gèrent l'allocation des ressources hospitalières, la planification des interventions et le suivi des traitements. Les systèmes de télémédecine utilisent des SMA pour coordonner les soins à distance et assurer le suivi continu des patients chroniques.

L'\textbf{industrie manufacturière} adopte les SMA pour créer des usines intelligentes où les machines, les robots et les systèmes de contrôle sont représentés par des agents qui coordonnent la production, optimisent l'utilisation des ressources et s'adaptent aux changements de demande. Cette approche permet une flexibilité et une efficacité accrues dans la production industrielle.

\section{Communication entre Agents}

\subsection{Langage de Communication entre Agents (ACL)}

La communication constitue le fondement de toute collaboration efficace entre agents dans un SMA. Le \textbf{Langage de Communication entre Agents (ACL - Agent Communication Language)} fournit un cadre standardisé permettant aux agents d'échanger des informations de manière structurée et compréhensible, indépendamment de leur implémentation interne ou de leur architecture.

Un ACL va bien au-delà d'un simple protocole de transmission de données. Il encapsule la \emph{sémantique} de la communication, définissant non seulement comment les messages sont structurés, mais aussi ce qu'ils signifient et quelles actions ils impliquent. Cette richesse sémantique permet aux agents de s'engager dans des interactions sophistiquées, allant de simples échanges d'informations à des négociations complexes et des coordinations élaborées.

Le standard le plus largement adopté est \textbf{FIPA-ACL} (Foundation for Intelligent Physical Agents - Agent Communication Language), qui définit une structure de message comprenant plusieurs composants essentiels. Le \emph{performatif} indique l'intention communicative du message (informer, demander, proposer, etc.). L'\emph{expéditeur} et le \emph{destinataire} identifient les agents impliqués dans la communication. Le \emph{contenu} porte l'information principale du message. Le \emph{langage de contenu} spécifie comment interpréter le contenu. L'\emph{ontologie} définit le vocabulaire et les concepts utilisés. Des paramètres additionnels comme l'\emph{identifiant de conversation}, le \emph{protocole} utilisé et les \emph{contraintes temporelles} enrichissent la communication.

Dans le contexte de Google ADK, l'ACL est implémenté de manière moderne et flexible, tirant parti des capacités des modèles de langage pour comprendre et générer des messages en langage naturel tout en maintenant la structure nécessaire pour une communication inter-agents fiable. Cette approche hybride combine la rigueur des ACL traditionnels avec la flexibilité et l'expressivité du langage naturel.

La standardisation de l'ACL apporte plusieurs avantages cruciaux. L'\emph{interopérabilité} permet à des agents développés indépendamment de communiquer efficacement. La \emph{réutilisabilité} facilite l'intégration de nouveaux agents dans des systèmes existants. La \emph{maintenabilité} est améliorée car les protocoles de communication sont clairement définis et documentés. L'\emph{extensibilité} permet d'ajouter de nouveaux types de messages et de protocoles selon les besoins évolutifs du système.

\subsection{Performatives FIPA-ACL (INFORM, REQUEST, QUERY, PROPOSE, etc.)}

Les \textbf{performatifs} représentent l'essence de la communication entre agents, définissant l'intention communicative derrière chaque message. Chaque performatif encode une action de communication spécifique avec sa propre sémantique, ses conditions de satisfaction et ses effets attendus sur l'état mental des agents participants.

Le performatif \textbf{INFORM} est utilisé lorsqu'un agent souhaite communiquer une information qu'il considère comme vraie à un autre agent. L'agent émetteur s'engage sur la véracité de l'information transmise et s'attend à ce que le récepteur mette à jour ses croyances en conséquence. Dans notre système agricole, l'Agent Météorologique utilise fréquemment INFORM pour notifier les autres agents des conditions météorologiques actuelles ou prévues. Par exemple, il pourrait envoyer un message INFORM contenant "La probabilité de pluie pour demain est de 80% avec une accumulation prévue de 15mm".

Le performatif \textbf{REQUEST} exprime une demande d'action de la part de l'agent émetteur. Il indique que l'émetteur souhaite que le destinataire effectue une action spécifique et s'attend à ce que cette action soit réalisée si le destinataire en a la capacité et la volonté. L'Agent Coordinateur Principal utilise REQUEST pour demander aux agents spécialisés d'analyser des aspects spécifiques d'une requête utilisateur. Par exemple, il pourrait envoyer "REQUEST: Analyser la rentabilité de la culture du maïs pour la saison prochaine" à l'Agent Économique.

Le performatif \textbf{QUERY} est employé pour interroger un autre agent sur une information spécifique. Contrairement à REQUEST qui demande une action, QUERY demande spécifiquement une information. Il existe plusieurs variantes de QUERY, notamment QUERY-IF pour demander si une proposition est vraie et QUERY-REF pour demander la valeur d'une expression. L'Agent Cultures pourrait utiliser "QUERY-IF: Est-ce que le sol de la parcelle Nord convient à la culture du cacao?" pour interroger l'Agent Ressources.

Le performatif \textbf{PROPOSE} initie une négociation en proposant une action ou un plan à un autre agent. Il indique que l'émetteur est prêt à effectuer une certaine action sous certaines conditions et attend une réponse du destinataire. Dans notre système, l'Agent Ressources pourrait proposer "PROPOSE: Réduire l'irrigation de 20% pour les deux prochaines semaines pour conserver l'eau" lors d'une période de sécheresse anticipée.

D'autres performatifs importants incluent \textbf{AGREE} pour accepter une proposition ou une demande, \textbf{REFUSE} pour décliner, \textbf{CONFIRM} pour confirmer une information incertaine, \textbf{DISCONFIRM} pour nier une information, et \textbf{SUBSCRIBE} pour s'abonner à des notifications d'événements spécifiques. Chaque performatif possède des conditions de satisfaction précises et des protocoles d'interaction associés qui garantissent une communication cohérente et prévisible entre agents.

\subsection{Protocoles d'interaction}

Les \textbf{protocoles d'interaction} définissent les séquences structurées d'échanges de messages entre agents pour accomplir des tâches spécifiques. Ces protocoles établissent les règles de conversation, spécifiant qui peut envoyer quel type de message à quel moment, et comment les agents doivent répondre dans différentes situations. Ils garantissent que les interactions complexes se déroulent de manière ordonnée et prévisible.

Le \textbf{protocole de requête simple} (Request Protocol) est l'un des plus fondamentaux. Il commence par un agent initiateur envoyant un REQUEST à un participant. Le participant peut répondre avec AGREE (indiquant qu'il accepte d'effectuer l'action), REFUSE (s'il ne peut ou ne veut pas effectuer l'action), ou NOT-UNDERSTOOD (s'il ne comprend pas la requête). Si le participant accepte, il effectue l'action demandée et envoie ensuite soit INFORM-DONE (action complétée avec succès) soit FAILURE (échec de l'action). Ce protocole simple mais efficace structure la majorité des interactions de demande-réponse dans notre système.

Le \textbf{protocole de négociation Contract Net} est particulièrement adapté pour la distribution de tâches et la sélection de fournisseurs de services. Un agent initiateur envoie un appel d'offres (CFP - Call For Proposals) à plusieurs participants potentiels. Les participants intéressés et capables répondent avec des PROPOSE contenant leurs offres. L'initiateur évalue les propositions et envoie ACCEPT-PROPOSAL au(x) meilleur(s) candidat(s) et REJECT-PROPOSAL aux autres. Les agents acceptés exécutent ensuite la tâche et rapportent les résultats. Dans notre système, ce protocole pourrait être utilisé lorsque l'Agent Coordinateur cherche le meilleur agent pour répondre à une requête spécifique.

Le \textbf{protocole de souscription} (Subscribe Protocol) permet aux agents de s'abonner à des notifications d'événements ou de changements d'état. Un agent envoie SUBSCRIBE avec les conditions de notification désirées. L'agent fournisseur répond avec AGREE ou REFUSE. Si accepté, le fournisseur envoie des messages INFORM chaque fois que les conditions spécifiées sont remplies. L'Agent Économique pourrait s'abonner aux mises à jour de prix du marché, recevant automatiquement des notifications lorsque les prix de certains produits agricoles changent significativement.

Les \textbf{protocoles de médiation} facilitent la communication entre agents qui ne peuvent pas interagir directement. Un agent médiateur reçoit des messages d'un agent source, les traite ou les traduit si nécessaire, et les transmet à l'agent destinataire. Dans notre système, l'Agent Coordinateur Principal agit souvent comme médiateur, traduisant les requêtes en langage naturel des utilisateurs en requêtes structurées pour les agents spécialisés.

L'implémentation de ces protocoles dans Google ADK bénéficie de la flexibilité des modèles de langage, permettant une interprétation plus nuancée des messages tout en maintenant la structure protocolaire nécessaire. Les agents peuvent ainsi gérer des variations dans la formulation des messages tout en respectant la sémantique des protocoles.

\subsection{Ontologies et représentation des connaissances}

Les \textbf{ontologies} dans les SMA fournissent un vocabulaire commun et une conceptualisation partagée du domaine d'application, permettant aux agents de communiquer avec précision et sans ambiguïté. Une ontologie définit les concepts, leurs propriétés, les relations entre concepts, et les contraintes qui gouvernent leur utilisation. Elle agit comme un dictionnaire sémantique partagé qui assure que tous les agents interprètent les informations de manière cohérente.

Dans notre système agricole, l'ontologie doit capturer la richesse et la complexité du domaine agricole camerounais. Les \emph{concepts fondamentaux} incluent les cultures (maïs, cacao, café, plantain, etc.), chacune avec ses propriétés spécifiques comme le cycle de croissance, les besoins en eau, la résistance aux maladies et les conditions optimales de culture. Les \emph{conditions environnementales} englobent les types de sol, les paramètres climatiques, les saisons et les zones agroclimatiques du Cameroun. Les \emph{pratiques agricoles} couvrent les techniques de culture, les méthodes d'irrigation, les traitements phytosanitaires et les calendriers agricoles.

Les \emph{relations} entre concepts enrichissent l'ontologie en capturant les dépendances et interactions du monde réel. Par exemple, la relation "convient\_à" lie un type de sol à une culture, "nécessite" connecte une culture à ses besoins en ressources, "traite" associe un produit phytosanitaire à une maladie. Ces relations permettent aux agents de raisonner sur le domaine et de dériver de nouvelles connaissances à partir des informations existantes.

La \emph{hiérarchie des concepts} organise les connaissances de manière structurée. Les cultures peuvent être organisées en familles botaniques, les maladies classées par type d'agent pathogène, les sols catégorisés selon leur composition et leurs propriétés. Cette organisation hiérarchique facilite le raisonnement par généralisation et spécialisation, permettant aux agents d'appliquer des connaissances générales à des cas spécifiques.

Les \emph{axiomes et règles} encodent les contraintes et les lois du domaine. Par exemple, "Une culture ne peut pas être semée si la température du sol est inférieure à son seuil minimal de germination" ou "L'irrigation doit être réduite pendant la période de maturation des fruits". Ces règles guident le comportement des agents et assurent la cohérence de leurs recommandations.

Dans Google ADK, l'intégration des ontologies avec les modèles de langage offre une approche unique. Les LLM peuvent comprendre et manipuler les concepts ontologiques exprimés en langage naturel tout en maintenant la rigueur sémantique nécessaire. Cette approche hybride permet une plus grande flexibilité dans l'expression des requêtes utilisateur tout en garantissant la précision des réponses des agents.

\section{Présentation de Google ADK (Agent Development Kit)}

\subsection{Qu'est-ce que Google ADK ?}

\textbf{Google Agent Development Kit (ADK)} représente une évolution majeure dans le développement de systèmes multi-agents, proposant une approche moderne qui tire parti des avancées récentes en intelligence artificielle, notamment les modèles de langage de grande taille. Contrairement aux frameworks traditionnels qui nécessitent une programmation explicite de chaque comportement d'agent, ADK permet de créer des agents intelligents en combinant la puissance des LLM avec une architecture d'agents structurée.

ADK est conçu pour simplifier radicalement le développement d'agents tout en offrant une flexibilité et une puissance exceptionnelles. Le framework permet aux développeurs de définir des agents en spécifiant leurs \emph{capacités}, leurs \emph{objectifs} et leurs \emph{contraintes} en langage naturel ou semi-structuré, laissant le modèle de langage sous-jacent gérer la complexité des interactions et du raisonnement. Cette approche déclarative contraste fortement avec l'approche impérative des frameworks traditionnels.

L'architecture d'ADK repose sur le concept d'\emph{agents augmentés par LLM}, où chaque agent combine une structure logique claire avec les capacités de compréhension et de génération du langage naturel. Cette combinaison permet aux agents de comprendre des requêtes complexes, de raisonner sur des informations non structurées, et de générer des réponses nuancées et contextuellement appropriées. Les agents ADK peuvent ainsi traiter une variété beaucoup plus large d'inputs et s'adapter à des situations non anticipées lors de leur conception.

La philosophie de conception d'ADK privilégie la \emph{simplicité d'utilisation} sans sacrifier la puissance. Les développeurs peuvent créer des agents fonctionnels avec quelques lignes de configuration, tout en ayant la possibilité de personnaliser profondément le comportement des agents pour des cas d'usage spécifiques. Cette approche progressive permet aux débutants de démarrer rapidement tout en offrant aux experts les outils nécessaires pour créer des systèmes sophistiqués.

L'intégration native avec l'écosystème Google Cloud constitue un autre avantage majeur d'ADK. Les agents peuvent facilement accéder aux services Google Cloud comme BigQuery pour l'analyse de données, Cloud Storage pour le stockage, et diverses API pour enrichir leurs capacités. Cette intégration transparente simplifie le développement d'agents qui nécessitent l'accès à des ressources externes ou le traitement de grandes quantités de données.

\subsection{Architecture et composants principaux}

L'architecture de Google ADK est conçue selon des principes de modularité et d'extensibilité, permettant aux développeurs de construire des systèmes complexes à partir de composants simples et réutilisables. Au cœur de cette architecture se trouve le \textbf{moteur d'exécution d'agents}, qui orchestre le cycle de vie des agents, gère leurs interactions et assure l'intégration avec les modèles de langage.

Le \textbf{Agent Core} constitue le composant fondamental de chaque agent ADK. Il encapsule l'identité de l'agent, ses capacités, ses objectifs et son état interne. Le Core gère également l'interface entre l'agent et le modèle de langage, traduisant les requêtes en prompts appropriés et interprétant les réponses du modèle dans le contexte de l'agent. Cette couche d'abstraction permet aux développeurs de se concentrer sur la logique métier plutôt que sur les détails techniques de l'interaction avec les LLM.

Le système de \textbf{Tools} (outils) représente l'un des aspects les plus puissants d'ADK. Les outils sont des fonctions ou des services que les agents peuvent invoquer pour étendre leurs capacités au-delà de la génération de texte. Un outil peut être aussi simple qu'une fonction de calcul ou aussi complexe qu'une API externe. Dans notre système agricole, nous définissons des outils pour accéder aux données météorologiques, consulter les bases de données agricoles, analyser les images de plantes, et calculer les indicateurs économiques. Le système de tools d'ADK gère automatiquement la découverte, l'invocation et la gestion des erreurs, simplifiant considérablement l'intégration de fonctionnalités externes.

Le \textbf{Context Manager} maintient et gère le contexte conversationnel et opérationnel de chaque agent. Il stocke l'historique des interactions, les informations de session, et tout état pertinent nécessaire pour maintenir la cohérence des conversations et des actions de l'agent. Le Context Manager implémente des stratégies sophistiquées de gestion de la mémoire, permettant aux agents de maintenir des conversations longues tout en optimisant l'utilisation des ressources.

Le \textbf{Orchestrator} coordonne les interactions entre multiple agents, gérant les flux de communication, la résolution des dépendances et l'ordonnancement des tâches. Dans notre système, l'Orchestrator permet à l'Agent Coordinateur Principal de solliciter efficacement les agents spécialisés, de gérer les réponses parallèles et de synthétiser les résultats. Il implémente également des mécanismes de gestion des erreurs et de récupération, assurant la robustesse du système face aux défaillances individuelles.

Le \textbf{Security Layer} assure la sécurité et la confidentialité des interactions. Il gère l'authentification des agents, l'autorisation des actions, le chiffrement des communications et l'audit des activités. Cette couche est particulièrement importante dans notre contexte agricole où les données des agriculteurs doivent être protégées et où l'accès aux différentes fonctionnalités doit être contrôlé selon les rôles et permissions.

\subsection{Modèles d'agents dans ADK}

Google ADK propose plusieurs modèles d'agents pré-configurés qui servent de points de départ pour différents types d'applications. Ces modèles encapsulent les meilleures pratiques et les patterns communs, permettant aux développeurs de démarrer rapidement tout en conservant la flexibilité de personnalisation.

Le modèle \textbf{Conversational Agent} est optimisé pour les interactions en langage naturel avec les utilisateurs. Il maintient le contexte conversationnel, gère les clarifications et les désambiguïsations, et génère des réponses naturelles et engageantes. Dans notre système, l'Agent Coordinateur Principal est basé sur ce modèle, lui permettant d'interagir naturellement avec les agriculteurs tout en comprenant leurs besoins complexes.

Le modèle \textbf{Task Agent} est conçu pour exécuter des tâches spécifiques avec efficacité et précision. Il se concentre sur l'accomplissement d'objectifs définis, utilisant les outils disponibles de manière optimale et rapportant les résultats de manière structurée. Nos agents spécialisés (Météorologique, Cultures, Santé des Plantes, Économique, Ressources) sont tous basés sur ce modèle, chacun étant configuré avec les outils et connaissances spécifiques à son domaine.

Le modèle \textbf{Analytical Agent} excelle dans l'analyse de données et la génération d'insights. Il peut traiter de grandes quantités d'informations, identifier des patterns, et produire des rapports détaillés. L'Agent Économique utilise des aspects de ce modèle pour analyser les tendances du marché, calculer la rentabilité des cultures et générer des recommandations financières basées sur des données complexes.

Le modèle \textbf{Monitoring Agent} est spécialisé dans la surveillance continue de systèmes ou de processus. Il détecte les anomalies, génère des alertes et peut déclencher des actions correctives. L'Agent Météorologique s'inspire de ce modèle pour surveiller en permanence les conditions climatiques et alerter les autres agents et les agriculteurs des changements significatifs ou des événements météorologiques importants.

Le modèle \textbf{Coordinator Agent} orchestre les activités d'autres agents, gérant les workflows complexes et assurant la cohérence des actions distribuées. Ce modèle implémente des stratégies sophistiquées de coordination, de résolution de conflits et d'optimisation des ressources. Notre Agent Coordinateur Principal utilise pleinement ce modèle pour gérer efficacement les interactions entre tous les agents spécialisés du système.

Chaque modèle d'agent dans ADK peut être étendu et personnalisé selon les besoins spécifiques. Les développeurs peuvent combiner des aspects de différents modèles, ajouter des comportements personnalisés, et intégrer des logiques métier spécifiques. Cette flexibilité permet de créer des agents parfaitement adaptés aux exigences uniques de chaque application.

\subsection{Intégration avec les LLM \textbf{Exemple:} Gemini}

L'intégration native avec les modèles de langage Gemini constitue l'une des caractéristiques les plus innovantes et puissantes de Google ADK. Cette intégration va bien au-delà d'une simple interface API, offrant une symbiose profonde entre l'architecture d'agents et les capacités des LLM modernes.

\textbf{Gemini}, le modèle de langage de pointe de Google, apporte aux agents ADK des capacités de compréhension et de génération du langage naturel sans précédent. Les agents peuvent comprendre des requêtes complexes formulées de manière naturelle, tenant compte du contexte, des nuances et même des implications non explicites. Cette compréhension sophistiquée permet aux agriculteurs d'interagir avec notre système comme ils le feraient avec un expert humain, sans avoir besoin d'apprendre des commandes spécifiques ou des interfaces complexes.

La \emph{génération contextuelle} permet aux agents de produire des réponses qui ne sont pas seulement correctes, mais aussi appropriées au contexte, au niveau de connaissance de l'utilisateur et à la situation spécifique. L'Agent Cultures, par exemple, peut expliquer les techniques de culture en adaptant son langage selon que l'utilisateur est un agriculteur expérimenté ou un débutant, fournissant plus ou moins de détails techniques selon le besoin.

L'\emph{apprentissage en contexte} (in-context learning) permet aux agents d'adapter leur comportement basé sur les exemples et les interactions précédentes sans nécessiter de réentraînement. Si un agriculteur utilise régulièrement des termes locaux ou des pratiques spécifiques à sa région, les agents apprennent progressivement à comprendre et utiliser ce vocabulaire, améliorant ainsi la qualité de la communication au fil du temps.

La capacité de \emph{raisonnement multi-étapes} de Gemini permet aux agents de décomposer des problèmes complexes en sous-problèmes, de planifier des séquences d'actions et de synthétiser des informations provenant de sources multiples. Lorsqu'un agriculteur demande "Quelle culture serait la plus rentable pour ma parcelle l'année prochaine?", l'agent peut orchestrer une analyse complexe impliquant les conditions du sol, les prévisions météorologiques, les tendances du marché et les ressources disponibles.

L'\emph{interprétation des outils} est grandement facilitée par Gemini, qui peut comprendre quand et comment utiliser les outils disponibles basé sur la requête de l'utilisateur. Le modèle peut également interpréter les résultats des outils et les intégrer naturellement dans ses réponses, créant une expérience transparente pour l'utilisateur. Si l'Agent Santé des Plantes utilise un outil d'analyse d'image pour diagnostiquer une maladie, Gemini peut expliquer les résultats en termes compréhensibles et proposer des actions concrètes.

La \emph{gestion multilingue} native de Gemini est particulièrement précieuse dans le contexte camerounais, permettant aux agents de communiquer en français, en anglais, et potentiellement dans les langues locales. Cette capacité assure que le système est accessible à tous les agriculteurs, indépendamment de leur langue préférée.

\section{Étude Comparative : Google ADK vs JADE}

\subsection{Tableau comparatif des caractéristiques}

Pour comprendre pleinement les différences et les similitudes entre Google ADK et JADE, il est essentiel d'examiner en détail leurs caractéristiques respectives. Cette comparaison vous aidera à comprendre pourquoi nous avons choisi ADK pour ce projet et comment les concepts que vous pourriez connaître de JADE se traduisent dans le nouveau framework.

\begin{table}[h]
\centering
\begin{tabular}{|p{4cm}|p{5.5cm}|p{5.5cm}|}
\hline
\textbf{Caractéristique} & \textbf{JADE} & \textbf{Google ADK} \\
\hline
\textbf{Langage de programmation} & Java & Python (principal), support multi-langage \\
\hline
\textbf{Architecture} & Basée sur conteneurs, architecture distribuée classique & Architecture cloud-native, intégration LLM native \\
\hline
\textbf{Communication entre agents} & FIPA-ACL strict, messages structurés & FIPA-ACL flexible + langage naturel via LLM \\
\hline
\textbf{Développement d'agents} & Programmation impérative, comportements explicites & Approche déclarative, comportements émergents via LLM \\
\hline
\textbf{Gestion du cycle de vie} & Manuelle via conteneurs et plateformes & Automatisée via orchestrateur cloud \\
\hline
\textbf{Scalabilité} & Limitée par l'architecture, scaling manuel & Cloud-native, auto-scaling intégré \\
\hline
\textbf{Interface utilisateur} & GUI Swing/AWT datée, développement séparé & Interfaces modernes web/mobile, intégration native \\
\hline
\textbf{Débogage} & Outils de débogage Java standard, sniffer JADE & Outils cloud modernes, logs structurés, tracing distribué \\
\hline
\textbf{Courbe d'apprentissage} & Raide, nécessite expertise Java et SMA & Plus douce grâce à l'approche déclarative \\
\hline
\textbf{Intégration IA} & Limitée, nécessite intégration manuelle & Native avec Gemini et autres modèles Google \\
\hline
\end{tabular}
\caption{Comparaison détaillée entre JADE et Google ADK}
\end{table}

Cette comparaison révèle des différences fondamentales dans la philosophie de conception. JADE, développé au début des années 2000, représente l'approche classique des SMA avec une emphase sur la conformité aux standards FIPA et le contrôle explicite du comportement des agents. Google ADK, en revanche, adopte une approche moderne qui tire parti des avancées en IA et en cloud computing pour simplifier le développement tout en augmentant les capacités.

\subsection{Avantages et inconvénients de chaque framework}

\textbf{JADE (Java Agent DEvelopment Framework)} a longtemps été le standard de facto pour le développement de systèmes multi-agents, et pour de bonnes raisons. Ses \emph{avantages} incluent une conformité stricte aux standards FIPA qui garantit l'interopérabilité avec d'autres systèmes conformes. La maturité du framework, avec plus de deux décennies de développement, signifie une base de code stable et bien testée. La large communauté d'utilisateurs a produit une documentation extensive, de nombreux exemples et des solutions à la plupart des problèmes communs. Le contrôle fin sur le comportement des agents permet d'implémenter des logiques complexes et des optimisations spécifiques.

Cependant, JADE présente également des \emph{inconvénients} significatifs dans le contexte moderne. La courbe d'apprentissage est raide, nécessitant une expertise approfondie en Java et en concepts SMA. Le développement est verbeux, nécessitant beaucoup de code boilerplate pour des fonctionnalités basiques. L'architecture montre son âge, avec des limitations en termes de scalabilité et d'intégration cloud. L'interface utilisateur basée sur Swing est datée et peu attrayante pour les utilisateurs modernes. L'intégration avec les technologies modernes d'IA nécessite un effort considérable.

\textbf{Google ADK} apporte une perspective fraîche avec ses propres \emph{avantages}. L'intégration native avec les LLM permet de créer des agents véritablement intelligents capables de comprendre et de générer du langage naturel. L'approche déclarative simplifie considérablement le développement, permettant de créer des agents fonctionnels avec peu de code. L'architecture cloud-native offre une scalabilité et une fiabilité exceptionnelles. Les outils de développement modernes, incluant le support pour Python et les notebooks Jupyter, facilitent le prototypage rapide. L'intégration transparente avec l'écosystème Google Cloud ouvre l'accès à une multitude de services puissants.

Les \emph{inconvénients} d'ADK incluent sa relative nouveauté, qui signifie une communauté plus petite et moins de ressources tierces. La dépendance à l'infrastructure cloud peut être problématique pour des déploiements on-premise ou dans des environnements déconnectés. Le coût d'utilisation des LLM peut devenir significatif pour des applications à grande échelle. La flexibilité de l'approche basée sur LLM peut parfois conduire à des comportements imprévisibles nécessitant une validation careful. Certains développeurs peuvent trouver l'abstraction du comportement des agents par les LLM moins transparente que l'approche explicite de JADE.

\subsection{Cas d'usage appropriés}

Le choix entre JADE et Google ADK dépend largement du contexte d'application, des contraintes techniques et des objectifs du projet. Comprendre les cas d'usage où chaque framework excelle permet de faire un choix éclairé.

\textbf{JADE} reste le choix approprié pour les \emph{systèmes industriels critiques} où la prédictibilité et le contrôle fin sont essentiels. Dans les environnements où chaque action doit être explicitement programmée et vérifiable, l'approche déterministe de JADE est préférable. Les \emph{systèmes embarqués} avec des ressources limitées bénéficient de l'empreinte relativement légère de JADE et de sa capacité à fonctionner sans connexion cloud. Les \emph{applications nécessitant une conformité stricte} aux standards FIPA pour l'interopérabilité avec des systèmes existants trouvent en JADE une solution éprouvée. Les \emph{projets académiques} étudiant les concepts fondamentaux des SMA peuvent préférer JADE pour sa transparence et son adhérence aux modèles théoriques classiques.

\textbf{Google ADK} excelle dans les \emph{applications orientées utilisateur} nécessitant des interactions en langage naturel. Notre système d'assistance agricole en est un exemple parfait, où les agriculteurs peuvent poser des questions complexes sans formation technique. Les \emph{systèmes nécessitant une adaptation rapide} à des domaines changeants bénéficient de la flexibilité des LLM pour comprendre de nouveaux concepts sans reprogrammation. Les \emph{applications d'analyse et de synthèse d'information} tirent parti des capacités de raisonnement et de génération des LLM. Les \emph{projets nécessitant une mise à l'échelle rapide} profitent de l'architecture cloud-native. Les \emph{systèmes multi-modaux} intégrant texte, images et autres données bénéficient de l'écosystème Google Cloud intégré.

Les \emph{applications hybrides} peuvent également être envisagées, utilisant JADE pour les composants critiques nécessitant un contrôle déterministe et ADK pour les interfaces utilisateur et les composants d'analyse. Cette approche permet de combiner les forces des deux frameworks selon les besoins spécifiques de chaque partie du système.

\subsection{Migration de concepts JADE vers ADK}

Pour les développeurs familiers avec JADE, la transition vers Google ADK nécessite de repenser certains concepts fondamentaux tout en s'appuyant sur les connaissances existantes des SMA. Cette section guide la traduction des concepts JADE vers leurs équivalents ADK.

Les \textbf{Agents JADE}, créés en étendant la classe Agent et implémentant des comportements spécifiques, se traduisent en ADK par des configurations d'agents augmentés par LLM. Au lieu d'écrire explicitement chaque comportement, vous définissez les capacités, objectifs et contraintes de l'agent, laissant le LLM générer les comportements appropriés. Par exemple, un agent JADE avec plusieurs CyclicBehaviours pour gérer différents types de messages devient en ADK un agent avec des tools et des prompts qui guident le LLM dans le traitement des requêtes.

Les \textbf{Behaviours JADE} (OneShotBehaviour, CyclicBehaviour, TickerBehaviour, etc.) n'ont pas d'équivalent direct en ADK car le modèle de programmation est fondamentalement différent. Au lieu de comportements explicites, ADK utilise des handlers d'événements et des tools que le LLM invoque selon le contexte. Un CyclicBehaviour qui vérifie périodiquement une condition devient en ADK une combinaison de triggers temporels et de logique conditionnelle gérée par l'orchestrateur.

Les \textbf{ACL Messages} structurés de JADE sont remplacés en ADK par une approche hybride. Bien que les agents ADK puissent échanger des messages structurés pour la compatibilité, ils excellent dans l'interprétation de messages en langage naturel. Un message JADE comme \textit{msg.setPerformative(ACLMessage.REQUEST); msg.setContent\\("temperature?");} peut simplement devenir "Quelle est la température actuelle?" en ADK, le LLM comprenant l'intention sans structure explicite.

Les \textbf{Conteneurs et Plateformes JADE} sont remplacés par l'infrastructure cloud d'ADK. La gestion manuelle des conteneurs, du Main Container et des agents containers devient automatique avec l'orchestrateur ADK. Le déploiement, qui nécessitait une configuration careful des hôtes et ports en JADE, devient une simple commande de déploiement cloud en ADK.

Le \textbf{Directory Facilitator (DF)} de JADE, utilisé pour la découverte de services, est remplacé en ADK par un système de registry plus flexible intégré à l'orchestrateur. Les agents n'ont plus besoin de s'enregistrer explicitement ; leurs capacités sont automatiquement découvertes et rendues disponibles aux autres agents.

Les \textbf{Ontologies JADE}, définies en Java avec des classes et des schémas stricts, évoluent en ADK vers des descriptions plus flexibles que le LLM peut interpréter. Au lieu de créer des classes Java pour chaque concept, vous pouvez décrire l'ontologie en langage naturel ou semi-structuré, permettant une évolution plus agile du domaine de connaissances.

Cette migration conceptuelle ne signifie pas l'abandon des principes fondamentaux des SMA. Au contraire, ADK permet d'implémenter ces principes de manière plus naturelle et flexible, réduisant la complexité technique tout en augmentant les capacités fonctionnelles. Les développeurs JADE trouveront que leurs connaissances des patterns d'interaction, des protocoles de coordination et des architectures multi-agents restent précieuses, même si leur implémentation technique diffère significativement.
\include{Chapters/PRÉSENTATION-DU-PROJET-AGRICULTURE-CAMEROUN}
\include{Chapters/ENVIRONNEMENT-DE-DÉVELOPPEMENT}
% \include{Chapters/IMPLÉMENTATION-AVEC-GOOGLE-ADK}
\chapter*{ INTÉGRATION ET DÉPLOIEMENT}
\addcontentsline{toc}{chapter}{ INTÉGRATION ET DÉPLOIEMENT}

\section{Interface Utilisateur}

\subsection{Interface web avec ADK}

Google ADK révolutionne la création d'interfaces utilisateur pour les systèmes multi-agents en fournissant une interface web moderne et réactive out-of-the-box. Cette interface, basée sur les dernières technologies web, offre une expérience utilisateur fluide et intuitive parfaitement adaptée aux besoins des agriculteurs camerounais.

\begin{figure}[h]
\centering
\framebox[0.9\textwidth]{
\parbox{0.85\textwidth}{
\centering
\textbf{Interface Web Agriculture Cameroun}\\[10pt]
% \includegraphics[width=0.9\textwidth]{[Interface principale montrant]}\\[5pt]
• Barre de chat intuitive en bas\\
• Historique des conversations à gauche\\
• Zone de réponse principale au centre\\
• Suggestions de questions fréquentes\\
• Indicateur d'agent actif en temps réel\\[10pt]
URL d'accès : \texttt{http://localhost:8000}\\
Responsive design adapté mobile/desktop
}
}
\caption{Interface web principale du système}
\end{figure}

L'interface web ADK est automatiquement générée lors du lancement du système avec la commande \texttt{adk web}. Cette approche zero-configuration permet de démarrer immédiatement sans configuration complexe d'interface.

\begin{figure}[h]
\centering
\begin{lstlisting}[language=Python, caption=Configuration de l'interface web ADK]
# Configuration dans agent.py pour personnaliser l'interface
# Le code complet est dans : agriculture/agent.py

# Configuration des métadonnées d'interface
INTERFACE_CONFIG = {
    "title": "Agriculture Cameroun - Assistant Intelligent",
    "description": "Votre conseiller agricole virtuel disponible 24/7",
    "theme": {
        "primary_color": "#2E7D32",  # Vert agriculture
        "secondary_color": "#FFC107",  # Jaune maïs
        "font_family": "Inter, sans-serif",
        "chat_width": "100%",
        "max_width": "1200px"
    },
    "welcome_message": """
    �� Bienvenue sur Agriculture Cameroun !

    Je suis votre assistant agricole intelligent. Je peux vous aider avec :
    - ��️ Prévisions météo et conseils climatiques
    - �� Recommandations de cultures et calendriers
    - �� Diagnostic de maladies et ravageurs
    - �� Analyse économique et prix du marché
    - �� Gestion optimale des ressources

    Posez-moi vos questions en français ou en anglais !
    """,
    "suggested_queries": [
        "Quand planter le maïs dans la région Centre ?",
        "Mon cacao a des taches brunes, que faire ?",
        "Prix actuel du café arabica au marché",
        "Comment économiser l'eau en saison sèche ?",
        "Prévisions météo pour Douala cette semaine"
    ],
    "features": {
        "voice_input": True,  # Activation entrée vocale
        "file_upload": True,  # Upload photos pour diagnostic
        "export_chat": True,  # Export des conversations
        "offline_mode": False,  # Mode hors ligne (futur)
        "multi_language": ["fr", "en"]  # Langues supportées
    }
}

# Personnalisation avancée de l'interface
class CustomWebInterface:
    """
    Extension de l'interface web ADK standard.
    Code complet : agriculture/interface/web_custom.py
    """

    def __init__(self):
        self.base_interface = adk.WebInterface(INTERFACE_CONFIG)
        self._add_custom_components()

    def _add_custom_components(self):
        """Ajoute des composants spécifiques à l'agriculture."""
        # Widget météo en temps réel
        self.base_interface.add_widget(
            WeatherWidget(position="top-right")
        )

        # Calendrier agricole interactif
        self.base_interface.add_widget(
            AgriculturalCalendarWidget(position="sidebar")
        )

        # Alertes et notifications
        self.base_interface.add_widget(
            AlertSystemWidget(
                types=["weather", "disease", "market"],
                position="top-banner"
            )
        )

    def render_response(self, response: str, metadata: Dict) -> str:
        """
        Personnalise le rendu des réponses.

        Ajoute :
        - Formatage spécial pour tableaux de données
        - Graphiques pour données numériques
        - Cartes pour informations géolocalisées
        - Icônes contextuelles
        """
        # Détection du type de contenu
        if "tableau" in response or "|" in response:
            return self._render_as_table(response)
        elif any(word in response for word in ["prix", "coût", "FCFA"]):
            return self._render_with_charts(response, metadata)
        elif "carte" in metadata.get("display_hints", []):
            return self._render_with_map(response, metadata)
        else:
            return self._render_standard(response)
\end{lstlisting}
\end{figure}

L'interface web intègre des fonctionnalités avancées spécifiquement conçues pour le contexte agricole camerounais. La \textbf{saisie vocale} permet aux agriculteurs ayant une alphabétisation limitée d'interagir naturellement avec le système. L'\textbf{upload d'images} facilite le diagnostic visuel des maladies et ravageurs. Le \textbf{mode sombre automatique} s'adapte aux conditions d'utilisation en extérieur.

\begin{figure}[h]
\centering
\framebox[0.9\textwidth]{
\parbox{0.85\textwidth}{
\centering
\textbf{Fonctionnalités Avancées de l'Interface}\\[10pt]
\textbf{1. Dashboard Agricole Personnalisé}\\
• Météo locale en temps réel\\
• Alertes personnalisées (maladies, météo, marché)\\
• Calendrier cultural interactif\\
• Suivi des activités agricoles\\[5pt]
\textbf{2. Mode Conversation Contextuelle}\\
• Historique persistant par session\\
• Suggestions basées sur le contexte\\
• Reprise de conversation après interruption\\
• Export PDF des recommandations\\[5pt]
\textbf{3. Visualisations Intelligentes}\\
• Graphiques de tendances de prix\\
• Cartes météo interactives\\
• Diagrammes de diagnostic\\
• Tableaux comparatifs de cultures
}
}
\caption{Capacités avancées de l'interface web}
\end{figure}

\subsection{API REST pour intégrations externes}

Au-delà de l'interface web, ADK expose automatiquement une API REST complète permettant l'intégration du système Agriculture Cameroun dans d'autres applications et services.

\begin{figure}[h]
\centering
\begin{lstlisting}[language=Python, caption=Points d'entrée de l'API REST]
# Configuration de l'API REST
# Fichier complet : agriculture/api/rest_api.py

from fastapi import FastAPI, HTTPException, Body
from pydantic import BaseModel, Field
from typing import Optional, List, Dict
import adk

# Modèles de données pour l'API
class QueryRequest(BaseModel):
    """Modèle pour les requêtes utilisateur."""
    query: str = Field(..., description="Question de l'utilisateur")
    session_id: Optional[str] = Field(None, description="ID de session")
    context: Optional[Dict] = Field(None, description="Contexte additionnel")
    language: str = Field("fr", description="Langue de réponse")
    region: Optional[str] = Field(None, description="Région spécifique")

class QueryResponse(BaseModel):
    """Modèle pour les réponses du système."""
    response: str = Field(..., description="Réponse générée")
    agents_consulted: List[str] = Field(..., description="Agents consultés")
    confidence: float = Field(..., description="Score de confiance")
    suggestions: List[str] = Field(..., description="Questions suggérées")
    metadata: Dict = Field(..., description="Métadonnées additionnelles")

# Initialisation de l'API
app = FastAPI(
    title="Agriculture Cameroun API",
    description="API REST pour l'assistant agricole intelligent",
    version="1.0.0",
    docs_url="/api/docs",  # Documentation Swagger
    redoc_url="/api/redoc"  # Documentation ReDoc
)

# Points d'entrée principaux
@app.post("/api/query", response_model=QueryResponse)
async def process_query(request: QueryRequest):
    """
    Traite une requête agricole.

    Exemples:
    - Prévisions météo: "Quelle est la météo à Yaoundé?"
    - Conseils cultures: "Quand planter le maïs?"
    - Diagnostic: "Feuilles jaunes sur tomates"
    """
    try:
        # Traitement de la requête
        result = await agriculture_system.process_query(
            query=request.query,
            session_id=request.session_id,
            context=request.context,
            language=request.language
        )

        return QueryResponse(
            response=result["response"],
            agents_consulted=result["agents_used"],
            confidence=result["confidence"],
            suggestions=result["follow_up_questions"],
            metadata=result["metadata"]
        )
    except Exception as e:
        raise HTTPException(status_code=500, detail=str(e))

@app.get("/api/weather/{region}")
async def get_weather_forecast(
    region: str,
    days: int = 7
):
    """Obtient les prévisions météo pour une région."""
    # Implémentation directe via l'agent météo
    pass

@app.post("/api/diagnose")
async def diagnose_plant_issue(
    crop: str = Body(...),
    symptoms: List[str] = Body(...),
    photos: Optional[List[str]] = Body(None)
):
    """Diagnostique un problème de culture."""
    # Utilisation de l'agent santé des plantes
    pass

@app.get("/api/market/prices")
async def get_market_prices(
    crop: Optional[str] = None,
    region: Optional[str] = None
):
    """Récupère les prix actuels du marché."""
    # Via l'agent économique
    pass

# Endpoints spécialisés pour intégrations
@app.post("/api/bulk/recommendations")
async def get_bulk_recommendations(
    farmers: List[Dict] = Body(...)
):
    """
    Génère des recommandations pour plusieurs agriculteurs.
    Utile pour les coopératives et organisations.
    """
    results = []
    for farmer in farmers:
        recommendation = await generate_personalized_plan(farmer)
        results.append(recommendation)
    return {"recommendations": results}

# Webhooks pour notifications
@app.post("/api/webhooks/weather-alerts")
async def setup_weather_webhook(
    callback_url: str = Body(...),
    regions: List[str] = Body(...),
    alert_types: List[str] = Body(...)
):
    """Configure des alertes météo automatiques."""
    # Configuration du système de notifications
    pass
\end{lstlisting}
\end{figure}

L'API REST suit les standards RESTful modernes avec documentation automatique via Swagger/OpenAPI. Chaque endpoint est optimisé pour des cas d'usage spécifiques, permettant une intégration flexible dans différents contextes.

\begin{figure}[h]
\centering
\framebox[0.9\textwidth]{
\parbox{0.85\textwidth}{
\centering
\textbf{Documentation Interactive de l'API}\\[10pt]
% \includegraphics[width=0.9\textwidth]{[Interface Swagger montrant :]}\\[5pt]
• Liste complète des endpoints\\
• Modèles de données avec exemples\\
• Interface de test "Try it out"\\
• Authentification via API key\\
• Codes de réponse détaillés\\[10pt]
Accès : \texttt{http://localhost:8000/api/docs}\\[5pt]
\textbf{Exemples de requêtes cURL :}\\[5pt]
\texttt{curl -X POST http://localhost:8000/api/query}\\
\texttt{-H "Content-Type: application/json"}\\
\texttt{-d '\{"query": "Météo Douala demain"\}'}
}
}
\caption{Interface de documentation Swagger de l'API}
\end{figure}

\subsection{Exemples d'utilisation}

Pour illustrer concrètement l'utilisation du système, examinons plusieurs scénarios réels d'interaction avec l'interface.

\begin{figure}[h]
\centering
\framebox[0.9\textwidth]{
\parbox{0.85\textwidth}{
\centering
\textbf{Scénario 1 : Planification de Culture}\\[10pt]
% \includegraphics[width=0.9\textwidth]{[Capture montrant :]}\\[5pt]
\textbf{Utilisateur :} "Je veux commencer la culture du cacao sur 2 hectares à Bafia"\\[5pt]
\textbf{Système :}
• Vérifie conditions climatiques de Bafia (Centre)\\
• Analyse aptitude des sols pour le cacao\\
• Propose calendrier cultural détaillé\\
• Estime investissement initial\\
• Suggère variétés adaptées\\[5pt]
\textit{Interface affiche carte interactive, calendrier et tableau financier}
}
}
\caption{Exemple de planification de nouvelle culture}
\end{figure}

\begin{figure}[h]
\centering
\begin{lstlisting}[language=JavaScript, caption=Intégration JavaScript de l'API]
// Exemple d'intégration dans une application mobile
// Code complet : examples/mobile_integration.js

class AgricultureCamerounClient {
    constructor(apiKey) {
        this.baseUrl = 'https://api.agriculture-cameroun.cm';
        this.apiKey = apiKey;
        this.sessionId = this.generateSessionId();
    }

    async askQuestion(query, context = {}) {
        const response = await fetch(`${this.baseUrl}/api/query`, {
            method: 'POST',
            headers: {
                'Content-Type': 'application/json',
                'X-API-Key': this.apiKey
            },
            body: JSON.stringify({
                query: query,
                session_id: this.sessionId,
                context: context,
                language: navigator.language.startsWith('fr') ? 'fr' : 'en'
            })
        });

        return await response.json();
    }

    async uploadDiseasePhoto(photo, crop, description) {
        // Conversion photo en base64
        const base64Photo = await this.convertToBase64(photo);

        const response = await fetch(`${this.baseUrl}/api/diagnose`, {
            method: 'POST',
            headers: {
                'Content-Type': 'application/json',
                'X-API-Key': this.apiKey
            },
            body: JSON.stringify({
                crop: crop,
                symptoms: this.extractSymptoms(description),
                photos: [base64Photo]
            })
        });

        const diagnosis = await response.json();
        return this.formatDiagnosisForDisplay(diagnosis);
    }

    // Utilisation avec notifications push
    async subscribeToAlerts(region, alertTypes) {
        const subscription = await this.registerPushNotifications();

        await fetch(`${this.baseUrl}/api/webhooks/alerts`, {
            method: 'POST',
            headers: {
                'Content-Type': 'application/json',
                'X-API-Key': this.apiKey
            },
            body: JSON.stringify({
                callback_url: subscription.endpoint,
                regions: [region],
                alert_types: alertTypes,
                user_preferences: {
                    quiet_hours: "22:00-06:00",
                    language: "fr",
                    urgency_threshold: "medium"
                }
            })
        });
    }
}

// Exemple d'utilisation dans une app React Native
const agriClient = new AgricultureCamerounClient('your-api-key');

// Conversation simple
const response = await agriClient.askQuestion(
    "Mes tomates ont des feuilles qui jaunissent par le bas"
);
console.log(response.response); // Diagnostic et recommandations

// Upload photo pour diagnostic
const diagnosis = await agriClient.uploadDiseasePhoto(
    photoFile,
    'tomate',
    'Jaunissement progressif des feuilles basales'
);

// Souscription aux alertes
await agriClient.subscribeToAlerts('Ouest', [
    'extreme_weather',
    'disease_outbreak',
    'market_opportunity'
]);
\end{lstlisting}
\end{figure}

\begin{figure}[h]
\centering
\framebox[0.9\textwidth]{
\parbox{0.85\textwidth}{
\centering
\textbf{Scénario 2 : Diagnostic Visuel par Photo}\\[10pt]
% \includegraphics[width=0.9\textwidth]{[Interface montrant :]}\\[5pt]
1. Bouton "�� Prendre une photo" prominent\\
2. Zone de preview de l'image uploadée\\
3. Analyse en temps réel avec indicateur de progression\\
4. Résultats structurés :\\
   • Maladie identifiée avec \% confiance\\
   • Zones affectées surlignées sur la photo\\
   • Plan de traitement étape par étape\\
   • Produits recommandés avec dosages\\[5pt]
\textit{Particulièrement utile pour diagnostics complexes}
}
}
\caption{Interface de diagnostic visuel des maladies}
\end{figure}

\section{Tests et Validation}

\subsection{Tests unitaires des agents}

La stratégie de test du système Agriculture Cameroun assure la fiabilité et la précision des conseils fournis aux agriculteurs. Chaque agent dispose de sa propre suite de tests couvrant ses fonctionnalités spécifiques.

\begin{figure}[h]
\centering
\begin{lstlisting}[language=Python, caption=Framework de tests unitaires pour agents ADK]
# Structure de tests dans tests/unit/test_agents.py
import pytest
from unittest.mock import Mock, patch
import adk
from agriculture.sub_agents.weather.agent import weather_agent
from agriculture.sub_agents.crops.agent import crops_agent

class TestWeatherAgent:
    """Tests unitaires pour l'agent météorologique."""

    @pytest.fixture
    def mock_weather_api(self):
        """Mock des appels API météo externes."""
        with patch('requests.get') as mock_get:
            mock_get.return_value.json.return_value = {
                "temperature": 28,
                "humidity": 75,
                "precipitation": 0,
                "forecast": "sunny"
            }
            yield mock_get

    def test_weather_forecast_accuracy(self, mock_weather_api):
        """Teste la précision des prévisions météo."""
        # Configuration du test
        query = "Météo pour Yaoundé demain"
        expected_keywords = ["température", "humidité", "ensoleillé"]

        # Exécution
        response = weather_agent.run(query)

        # Vérifications
        assert response.success
        assert all(keyword in response.content.lower()
                  for keyword in expected_keywords)
        assert "28" in response.content  # Température
        assert mock_weather_api.called

    def test_agricultural_alerts(self):
        """Teste la génération d'alertes agricoles."""
        # Simulation conditions extrêmes
        extreme_conditions = {
            "temperature": 40,
            "humidity": 30,
            "wind_speed": 45
        }

        with patch('agriculture.tools.get_weather_data',
                  return_value=extreme_conditions):
            response = weather_agent.run(
                "Conditions pour pulvérisation aujourd'hui?"
            )

            # Doit déconseiller la pulvérisation
            assert "déconseillé" in response.content.lower()
            assert "vent" in response.content.lower()
            assert "température" in response.content.lower()

    @pytest.mark.parametrize("region,expected_pattern", [
        ("Nord", "saison sèche|harmattan"),
        ("Littoral", "pluie|humidité élevée"),
        ("Ouest", "altitude|fraîcheur"),
    ])
    def test_regional_specificity(self, region, expected_pattern):
        """Teste l'adaptation aux spécificités régionales."""
        import re

        query = f"Climat typique de la région {region}"
        response = weather_agent.run(query)

        assert re.search(expected_pattern, response.content, re.IGNORECASE)

class TestCropsAgent:
    """Tests unitaires pour l'agent cultures."""

    def test_crop_calendar_generation(self):
        """Teste la génération de calendriers culturaux."""
        query = "Calendrier cultural du maïs pour Garoua"
        response = crops_agent.run(query)

        # Vérification des éléments essentiels
        calendar_elements = [
            "préparation du sol",
            "semis",
            "sarclage",
            "fertilisation",
            "récolte"
        ]

        assert all(element in response.content.lower()
                  for element in calendar_elements)

        # Vérification des périodes spécifiques à Garoua
        assert any(month in response.content
                  for month in ["mai", "juin"])  # Période de semis

    def test_crop_recommendations_constraints(self):
        """Teste les recommandations avec contraintes."""
        query = """
        Quelle culture pour :
        - Sol argileux pH 5.5
        - Budget limité (100,000 FCFA/ha)
        - Main d'œuvre familiale seulement
        - Région Centre
        """

        response = crops_agent.run(query)

        # Doit recommander cultures adaptées
        assert "manioc" in response.content.lower() or \
               "plantain" in response.content.lower()

        # Doit mentionner l'adaptation au sol acide
        assert "pH" in response.content or "acide" in response.content

        # Doit considérer le budget
        assert "budget" in response.content.lower()

# Tests d'intégration des outils
class TestAgentTools:
    """Tests des outils utilisés par les agents."""

    def test_soil_analysis_tool(self):
        """Teste l'outil d'analyse du sol."""
        from agriculture.sub_agents.resources.tools import analyze_soil_data

        result = analyze_soil_data(
            ph=6.5,
            organic_matter_percent=2.5,
            nitrogen_ppm=15,
            phosphorus_ppm=25,
            potassium_ppm=180,
            texture="loamy",
            crop_planned="cacao"
        )

        assert "soil_health_score" in result
        assert 60 <= result["soil_health_score"] <= 80
        assert "recommendations" in result
        assert len(result["recommendations"]) > 0

    def test_disease_diagnostic_tool(self):
        """Teste l'outil de diagnostic des maladies."""
        from agriculture.sub_agents.health.tools import diagnose_plant_disease

        diagnosis = diagnose_plant_disease(
            crop="tomate",
            symptoms=["feuilles jaunes", "taches brunes", "flétrissement"],
            affected_parts=["feuilles", "tige"],
            development_stage="floraison"
        )

        assert "possible_diseases" in diagnosis
        assert len(diagnosis["possible_diseases"]) > 0
        assert diagnosis["possible_diseases"][0]["probability"] > 50
        assert "treatment_plan" in diagnosis

# Tests de performance
class TestPerformance:
    """Tests de performance du système."""

    @pytest.mark.benchmark
    def test_response_time(self, benchmark):
        """Teste le temps de réponse des agents."""
        query = "Quand planter le maïs à Yaoundé?"

        # Benchmark de la performance
        result = benchmark(weather_agent.run, query)

        assert result.success
        # Temps de réponse acceptable < 2 secondes
        assert benchmark.stats['mean'] < 2.0

    def test_concurrent_requests(self):
        """Teste la gestion de requêtes concurrentes."""
        import asyncio
        from concurrent.futures import ThreadPoolExecutor

        queries = [
            "Météo Douala",
            "Prix du cacao",
            "Maladies du maïs",
            "Calendrier cultural café",
            "Analyse sol argileux"
        ]

        with ThreadPoolExecutor(max_workers=5) as executor:
            futures = [executor.submit(process_query, q) for q in queries]
            results = [f.result() for f in futures]

        assert all(r is not None for r in results)
        assert len(results) == len(queries)
\end{lstlisting}
\end{figure}

Les tests unitaires couvrent non seulement la fonctionnalité correcte mais aussi la \textbf{pertinence agricole} des réponses. Des tests spécifiques vérifient que les recommandations sont adaptées au contexte camerounais et suivent les bonnes pratiques agricoles locales.

\subsection{Tests d'intégration}

Les tests d'intégration vérifient que les différents composants du système fonctionnent harmonieusement ensemble, particulièrement important pour un système multi-agents où la coordination est cruciale.

\begin{figure}[h]
\centering
\begin{lstlisting}[language=Python, caption=Suite de tests d'intégration]
# Tests d'intégration dans tests/integration/test_system.py
import pytest
from agriculture import main_agent
import time

class TestMultiAgentIntegration:
    """Tests d'intégration du système complet."""

    @pytest.fixture(scope="class")
    def system_instance(self):
        """Instance du système pour les tests."""
        # Configuration de test
        test_config = {
            "default_region": "Centre",
            "default_language": "fr",
            "test_mode": True
        }
        return main_agent.MainCoordinatorAgent(test_config)

    def test_complex_query_routing(self, system_instance):
        """Teste le routage de requêtes complexes."""
        # Requête nécessitant plusieurs agents
        complex_query = """
        Je veux planter du maïs le mois prochain à Bafoussam.
        Quelle est la météo prévue et quel budget prévoir?
        Y a-t-il des risques de maladies en cette période?
        """

        response = system_instance.process_query(complex_query)

        # Vérifications multi-agents
        assert "météo" in response.lower() or "pluie" in response.lower()
        assert "budget" in response.lower() or "coût" in response.lower()
        assert "maladie" in response.lower() or "risque" in response.lower()

        # Vérifier cohérence temporelle
        assert "mois prochain" in response.lower() or \
               any(month in response.lower()
                   for month in ["avril", "mai", "juin"])

    def test_context_preservation(self, system_instance):
        """Teste la préservation du contexte entre requêtes."""
        session_id = "test_session_123"

        # Première requête
        response1 = system_instance.process_query(
            "Je cultive du cacao dans l'Ouest",
            session_id=session_id
        )

        # Deuxième requête utilisant contexte implicite
        response2 = system_instance.process_query(
            "Quelles sont les maladies courantes?",
            session_id=session_id
        )

        # Doit mentionner maladies du cacao, pas général
        assert "cacao" in response2.lower()
        assert any(disease in response2.lower()
                  for disease in ["pourriture", "swollen shoot"])

    def test_error_recovery(self, system_instance):
        """Teste la récupération d'erreurs."""
        # Simulation d'erreur d'un sous-agent
        with patch('agriculture.sub_agents.weather.agent.run',
                  side_effect=Exception("API météo indisponible")):

            response = system_instance.process_query(
                "Météo et conseils pour planter demain"
            )

            # Doit quand même fournir conseils cultures
            assert "planter" in response.lower()
            # Doit mentionner problème météo
            assert "météo" in response.lower()
            assert "non disponible" in response.lower() or \
                   "problème" in response.lower()

class TestEndToEndScenarios:
    """Tests de scénarios complets bout-en-bout."""

    def test_new_farmer_onboarding(self, system_instance):
        """Teste le parcours d'un nouvel agriculteur."""
        session_id = "new_farmer_001"

        # Étape 1: Présentation
        intro = system_instance.process_query(
            "Bonjour, je suis nouveau dans l'agriculture",
            session_id=session_id
        )
        assert "bienvenue" in intro.lower()
        assert any(word in intro.lower()
                  for word in ["aide", "assister", "conseiller"])

        # Étape 2: Contexte
        context_response = system_instance.process_query(
            "J'ai 2 hectares à Dschang, sol volcanique",
            session_id=session_id,
            user_context={"region": "Ouest", "land_size": 2}
        )
        assert "volcanique" in context_response.lower()
        assert "Dschang" in context_response or "Ouest" in context_response

        # Étape 3: Recommandation culturale
        crop_advice = system_instance.process_query(
            "Quelle culture me conseillez-vous?",
            session_id=session_id
        )
        # Doit recommander cultures adaptées à l'Ouest
        assert any(crop in crop_advice.lower()
                  for crop in ["café", "pomme de terre", "maraîcher"])

    def test_seasonal_advisory_flow(self):
        """Teste le flux de conseil saisonnier."""
        # Obtenir saison actuelle
        current_month = time.strftime("%B")
        current_season = get_current_season()

        # Requête contextuelle à la saison
        seasonal_query = f"Que faire ce mois de {current_month}?"
        response = system_instance.process_query(seasonal_query)

        # Doit mentionner activités saisonnières
        if current_season == "planting":
            assert any(word in response.lower()
                      for word in ["semis", "planter", "préparer"])
        elif current_season == "growing":
            assert any(word in response.lower()
                      for word in ["entretien", "sarclage", "fertilisation"])
        elif current_season == "harvest":
            assert any(word in response.lower()
                      for word in ["récolte", "stockage", "vente"])

# Tests de validation des données
class TestDataValidation:
    """Valide l'exactitude des données agricoles."""

    def test_crop_calendar_accuracy(self):
        """Vérifie l'exactitude des calendriers culturaux."""
        from agriculture.utils.data import CROP_CALENDARS

        # Vérifier cohérence des données
        for crop, calendar in CROP_CALENDARS.items():
            for region, timing in calendar.items():
                # Les périodes doivent être valides
                assert 1 <= timing["planting_month"] <= 12
                assert timing["growth_duration"] > 0
                assert timing["growth_duration"] < 365

                # Logique agricole
                if crop == "maïs":
                    # Le maïs a un cycle de 90-120 jours
                    assert 90 <= timing["growth_duration"] <= 120

    def test_price_data_reasonableness(self):
        """Vérifie la cohérence des données de prix."""
        from agriculture.utils.data import MARKET_PRICES

        for crop, prices in MARKET_PRICES.items():
            # Les prix doivent être positifs et raisonnables
            assert prices["min"] > 0
            assert prices["max"] > prices["min"]
            assert prices["average"] > prices["min"]
            assert prices["average"] < prices["max"]

            # Vérification de cohérence par culture
            if crop == "cacao":
                # Prix du cacao en FCFA/kg
                assert 800 <= prices["average"] <= 2000
\end{lstlisting}
\end{figure}

\subsection{Scénarios de test complets}

Les scénarios de test complets simulent des cas d'usage réels pour valider que le système répond correctement aux besoins des agriculteurs camerounais.

\begin{figure}[h]
\centering
\framebox[0.9\textwidth]{
\parbox{0.85\textwidth}{
\centering
\textbf{Scénarios de Test Automatisés}\\[10pt]
�� Localisation : \texttt{tests/scenarios/}\\[10pt]
\textbf{Scénarios critiques testés :}\\[5pt]
1. \textbf{Urgence Phytosanitaire}\\
   • Détection rapide invasion ravageurs\\
   • Mobilisation multi-agents\\
   • Plan d'action immédiat\\[5pt]
2. \textbf{Planification Saisonnière}\\
   • Analyse conditions initiales\\
   • Recommandations coordonnées\\
   • Suivi sur cycle complet\\[5pt]
3. \textbf{Optimisation Économique}\\
   • Analyse rentabilité multi-cultures\\
   • Adaptation aux prix marché\\
   • Stratégies de diversification\\[5pt]
4. \textbf{Adaptation Climatique}\\
   • Réponse aux alertes météo\\
   • Ajustement des pratiques\\
   • Résilience long terme
}
}
\caption{Scénarios de test end-to-end}
\end{figure}

\section{Déploiement}

\subsection{Déploiement local}

Le déploiement local du système Agriculture Cameroun est conçu pour être simple et rapide, permettant aux développeurs et testeurs de démarrer immédiatement.

\begin{figure}[h]
\centering
\begin{lstlisting}[language=bash, caption=Script de déploiement local automatisé]
#!/bin/bash
# Script de déploiement local : scripts/deploy_local.sh

echo "�� Déploiement local d'Agriculture Cameroun"
echo "=========================================="

# Vérification des prérequis
check_requirements() {
    echo "Vérification des prérequis..."

    # Python 3.12+
    if ! python3 --version | grep -E "3\.(1[2-9]|[2-9][0-9])" > /dev/null; then
        echo "❌ Python 3.12+ requis"
        exit 1
    fi

    # Poetry
    if ! command -v poetry &> /dev/null; then
        echo "�� Installation de Poetry..."
        curl -sSL https://install.python-poetry.org | python3 -
        export PATH="$HOME/.local/bin:$PATH"
    fi

    # Git
    if ! command -v git &> /dev/null; then
        echo "❌ Git requis pour le déploiement"
        exit 1
    fi

    echo "✅ Tous les prérequis sont satisfaits"
}

# Installation du projet
install_project() {
    echo -e "\n�� Installation du projet..."

    # Clone ou mise à jour
    if [ -d "agriculture-cameroun" ]; then
        cd agriculture-cameroun
        git pull origin main
    else
        git clone https://github.com/Nameless0l/agriculture-cameroun.git
        cd agriculture-cameroun
    fi

    # Installation des dépendances
    echo "�� Installation des dépendances Python..."
    poetry install --no-interaction --verbose

    # Configuration de l'environnement
    if [ ! -f ".env" ]; then
        echo -e "\n⚙️ Configuration de l'environnement..."
        cp .env.example .env

        # Demander la clé API Gemini
        read -p "Entrez votre clé API Gemini: " gemini_key
        sed -i "s/your_gemini_api_key_here/$gemini_key/" .env

        # Configuration régionale
        echo "Sélectionnez votre région par défaut:"
        select region in "Centre" "Littoral" "Ouest" "Nord" "Sud" "Est"; do
            sed -i "s/DEFAULT_REGION=.*/DEFAULT_REGION=$region/" .env
            break
        done
    fi
}

# Lancement du système
start_system() {
    echo -e "\n�� Lancement du système..."

    # Activation de l'environnement Poetry
    poetry shell

    # Vérification de la configuration
    python scripts/check_config.py
    if [ $? -ne 0 ]; then
        echo "❌ Erreur de configuration"
        exit 1
    fi

    # Lancement avec ADK
    echo -e "\n✨ Démarrage d'Agriculture Cameroun..."
    echo "�� Interface web : http://localhost:8000"
    echo "�� Documentation API : http://localhost:8000/api/docs"
    echo -e "\nAppuyez sur Ctrl+C pour arrêter le serveur\n"

    # Lancement avec options de développement
    ADK_DEV_MODE=true ADK_LOG_LEVEL=INFO adk web --port 8000 --reload
}

# Menu principal
main_menu() {
    clear
    echo "�� Agriculture Cameroun - Déploiement Local"
    echo "==========================================="
    echo "1. Installation complète (première fois)"
    echo "2. Mise à jour et lancement"
    echo "3. Lancement seulement"
    echo "4. Tests du système"
    echo "5. Quitter"

    read -p "Choisissez une option (1-5): " choice

    case $choice in
        1)
            check_requirements
            install_project
            start_system
            ;;
        2)
            cd agriculture-cameroun
            git pull origin main
            poetry install
            start_system
            ;;
        3)
            cd agriculture-cameroun
            start_system
            ;;
        4)
            cd agriculture-cameroun
            poetry run pytest tests/ -v
            ;;
        5)
            echo "Au revoir ! ��"
            exit 0
            ;;
        *)
            echo "Option invalide"
            sleep 2
            main_menu
            ;;
    esac
}

# Point d'entrée
main_menu
\end{lstlisting}
\end{figure}

Le déploiement local inclut des \textbf{outils de développement avancés} comme le rechargement automatique du code, les logs détaillés et l'accès aux outils de débogage ADK. Le mode développement active également des endpoints supplémentaires pour tester individuellement chaque agent.

\subsection{Containerisation avec Docker}

La containerisation Docker assure une portabilité parfaite et simplifie grandement le déploiement en production.

\begin{figure}[h]
\centering
\begin{lstlisting}[language=Dockerfile, caption=Docker optimisé pour Agriculture Cameroun]
# Dockerfile multi-stage pour optimisation
FROM python:3.12-slim as builder

# Variables d'environnement pour optimisation
ENV PYTHONUNBUFFERED=1 \
    PYTHONDONTWRITEBYTECODE=1 \
    POETRY_VERSION=1.7.0 \
    POETRY_HOME="/opt/poetry" \
    POETRY_VIRTUALENVS_IN_PROJECT=true \
    POETRY_NO_INTERACTION=1

# Installation de Poetry
RUN apt-get update && apt-get install -y \
    curl \
    build-essential \
    && curl -sSL https://install.python-poetry.org | python3 - \
    && apt-get clean \
    && rm -rf /var/lib/apt/lists/*

ENV PATH="$POETRY_HOME/bin:$PATH"

# Copie des fichiers de dépendances
WORKDIR /app
COPY pyproject.toml poetry.lock ./

# Installation des dépendances
RUN poetry install --no-root --no-dev

# Stage de production
FROM python:3.12-slim

ENV PYTHONUNBUFFERED=1 \
    PYTHONDONTWRITEBYTECODE=1

# Création utilisateur non-root
RUN useradd -m -u 1000 agriuser && \
    mkdir -p /app && \
    chown -R agriuser:agriuser /app

WORKDIR /app

# Copie des dépendances depuis builder
COPY --from=builder /app/.venv /app/.venv
ENV PATH="/app/.venv/bin:$PATH"

# Copie du code application
COPY --chown=agriuser:agriuser . .

# Changement vers utilisateur non-root
USER agriuser

# Healthcheck
HEALTHCHECK --interval=30s --timeout=10s --start-period=5s --retries=3 \
    CMD curl -f http://localhost:8000/health || exit 1

# Port d'exposition
EXPOSE 8000

# Commande de démarrage
CMD ["adk", "web", "--host", "0.0.0.0", "--port", "8000"]
\end{lstlisting}
\end{figure}

\begin{figure}[h]
\centering
\begin{lstlisting}[language=yaml, caption=Docker Compose pour environnement complet]
# docker-compose.yml
version: '3.8'

services:
  # Service principal Agriculture Cameroun
  agriculture-api:
    build:
      context: .
      dockerfile: Dockerfile
    image: agriculture-cameroun:latest
    container_name: agriculture_cameroun_api
    ports:
      - "8000:8000"
    environment:
      - GEMINI_API_KEY=${GEMINI_API_KEY}
      - DEFAULT_REGION=${DEFAULT_REGION:-Centre}
      - DEFAULT_LANGUAGE=${DEFAULT_LANGUAGE:-fr}
      - LOG_LEVEL=${LOG_LEVEL:-INFO}
      - REDIS_URL=redis://redis:6379/0
    volumes:
      - ./data:/app/data
      - ./logs:/app/logs
    depends_on:
      - redis
    restart: unless-stopped
    networks:
      - agriculture-network

  # Cache Redis pour performances
  redis:
    image: redis:7-alpine
    container_name: agriculture_redis
    ports:
      - "6379:6379"
    volumes:
      - redis-data:/data
    command: redis-server --appendonly yes
    restart: unless-stopped
    networks:
      - agriculture-network

  # Monitoring avec Prometheus
  prometheus:
    image: prom/prometheus:latest
    container_name: agriculture_prometheus
    ports:
      - "9090:9090"
    volumes:
      - ./monitoring/prometheus.yml:/etc/prometheus/prometheus.yml
      - prometheus-data:/prometheus
    command:
      - '--config.file=/etc/prometheus/prometheus.yml'
      - '--storage.tsdb.path=/prometheus'
    restart: unless-stopped
    networks:
      - agriculture-network

  # Dashboard Grafana
  grafana:
    image: grafana/grafana:latest
    container_name: agriculture_grafana
    ports:
      - "3000:3000"
    environment:
      - GF_SECURITY_ADMIN_PASSWORD=${GRAFANA_PASSWORD:-admin}
    volumes:
      - grafana-data:/var/lib/grafana
      - ./monitoring/grafana/dashboards:/etc/grafana/provisioning/dashboards
    depends_on:
      - prometheus
    restart: unless-stopped
    networks:
      - agriculture-network

  # Nginx reverse proxy
  nginx:
    image: nginx:alpine
    container_name: agriculture_nginx
    ports:
      - "80:80"
      - "443:443"
    volumes:
      - ./nginx/nginx.conf:/etc/nginx/nginx.conf
      - ./nginx/ssl:/etc/nginx/ssl
    depends_on:
      - agriculture-api
    restart: unless-stopped
    networks:
      - agriculture-network

volumes:
  redis-data:
  prometheus-data:
  grafana-data:

networks:
  agriculture-network:
    driver: bridge
\end{lstlisting}
\end{figure}

La configuration Docker Compose fournit un environnement de production complet avec cache Redis pour les performances, monitoring Prometheus/Grafana pour l'observabilité, et Nginx pour la terminaison SSL et le load balancing.

\subsection{Déploiement en production}

Le déploiement en production nécessite des considérations supplémentaires pour assurer la fiabilité, la sécurité et la scalabilité du système.

\begin{figure}[h]
\centering
\framebox[0.9\textwidth]{
\parbox{0.85\textwidth}{
\centering
\textbf{Architecture de Production}\\[10pt]
% \includegraphics[width=0.9\textwidth]{[Diagramme montrant :]}\\[5pt]
• Load Balancer (Nginx/HAProxy)\\
• Cluster ADK multi-instances\\
• Cache distribué Redis\\
• Base de données PostgreSQL\\
• Storage object pour médias\\
• CDN pour assets statiques\\
• Monitoring stack complet\\[10pt]
\textbf{Plateformes Cloud Supportées :}\\
✓ Google Cloud Platform (recommandé)\\
✓ AWS (EC2, ECS, Lambda)\\
✓ Azure (Container Instances)\\
✓ DigitalOcean (Apps Platform)\\
✓ Serveurs dédiés on-premise
}
}
\caption{Architecture de déploiement production}
\end{figure}

\begin{figure}[h]
\centering
\begin{lstlisting}[language=bash, caption=Script de déploiement production (GCP)]
#!/bin/bash
# deploy_production_gcp.sh - Déploiement sur Google Cloud

# Configuration
PROJECT_ID="agriculture-cameroun-prod"
REGION="europe-west1"
SERVICE_NAME="agriculture-api"
IMAGE_NAME="gcr.io/$PROJECT_ID/agriculture-cameroun"

# Build et push de l'image Docker
echo "��️ Construction de l'image Docker..."
docker build -t $IMAGE_NAME:latest .
docker push $IMAGE_NAME:latest

# Déploiement sur Cloud Run
echo "�� Déploiement sur Cloud Run..."
gcloud run deploy $SERVICE_NAME \
    --image $IMAGE_NAME:latest \
    --platform managed \
    --region $REGION \
    --allow-unauthenticated \
    --min-instances 2 \
    --max-instances 100 \
    --memory 2Gi \
    --cpu 2 \
    --set-env-vars "DEFAULT_REGION=Centre,DEFAULT_LANGUAGE=fr" \
    --set-secrets "GEMINI_API_KEY=gemini-api-key:latest"

# Configuration du domaine personnalisé
echo "�� Configuration du domaine..."
gcloud run domain-mappings create \
    --service $SERVICE_NAME \
    --domain agriculture-cameroun.cm \
    --region $REGION

# Mise en place du monitoring
echo "�� Configuration du monitoring..."
gcloud monitoring dashboards create \
    --config-from-file=monitoring/dashboard-config.yaml

# Configuration des alertes
gcloud alpha monitoring policies create \
    --notification-channels=$ALERT_CHANNEL \
    --display-name="Agriculture API Alerts" \
    --condition-threshold-value=0.95 \
    --condition-threshold-duration=300s

echo "✅ Déploiement terminé!"
echo "�� URL: https://agriculture-cameroun.cm"
\end{lstlisting}
\end{figure}

Le déploiement en production intègre des \textbf{bonnes pratiques DevOps} essentielles incluant l'infrastructure as code (Terraform), CI/CD automatisé (GitHub Actions), stratégies de rollback automatique, tests de charge et monitoring, sauvegardes automatiques et plans de disaster recovery.

\begin{figure}[h]
\centering
\framebox[0.9\textwidth]{
\parbox{0.85\textwidth}{
\centering
\textbf{Checklist de Production}\\[10pt]
✓ \textbf{Sécurité}\\
• HTTPS obligatoire avec certificats SSL\\
• API keys avec rotation régulière\\
• Rate limiting et DDoS protection\\
• Audit logs et monitoring sécurité\\[5pt]
✓ \textbf{Performance}\\
• Cache multi-niveaux (CDN, Redis, local)\\
• Optimisation des requêtes LLM\\
• Compression et minification\\
• Database connection pooling\\[5pt]
✓ \textbf{Fiabilité}\\
• Health checks et auto-healing\\
• Circuit breakers pour dépendances\\
• Graceful degradation\\
• Backups automatiques quotidiens\\[5pt]
✓ \textbf{Observabilité}\\
• Logs centralisés (ELK/Datadog)\\
• Métriques détaillées (Prometheus)\\
• Tracing distribué (Jaeger)\\
• Alerting intelligent
}
}
\caption{Exigences pour déploiement production}
\end{figure}

Cette architecture de production garantit que le système Agriculture Cameroun peut servir des milliers d'agriculteurs simultanément avec une haute disponibilité et des performances optimales, tout en maintenant la sécurité et la facilité de maintenance.

\chapter*{Conclusion}

\section*{Récapitulatif des concepts clés}

Au terme de ce tutoriel, nous avons exploré en profondeur l'implémentation d'un système multi-agents moderne utilisant Google ADK pour répondre aux défis agricoles au Cameroun. Cette approche révolutionnaire démontre comment l'intelligence artificielle distribuée peut transformer l'agriculture traditionnelle en agriculture intelligente.

Sur le plan théorique, nous avons maîtrisé l'architecture des systèmes multi-agents en comprenant les interactions complexes entre agents autonomes, réactifs et pro-actifs évoluant dans un environnement partagé. La communication inter-agents s'est révélée centrale, nécessitant une maîtrise approfondie des protocoles d'échange d'informations et des mécanismes de coordination sophistiqués. La spécialisation des agents a permis de créer des experts virtuels dans des domaines spécifiques tels que la météorologie, les cultures, la santé des plantes, l'économie et la gestion des ressources. L'intégration des modèles de langage de grande taille (LLM) a considérablement enrichi les capacités de raisonnement et de communication naturelle des agents.

D'un point de vue technique, la maîtrise de Google ADK s'est avérée fondamentale, depuis l'installation et la configuration jusqu'à l'utilisation avancée du framework. Le développement d'agents spécialisés adaptés au contexte agricole camerounais a nécessité une compréhension fine des besoins locaux et des contraintes environnementales. L'intégration d'APIs externes pour les données météorologiques et les marchés agricoles a ouvert de nouvelles possibilités d'analyse en temps réel. Le développement d'interfaces utilisateur intuitives a permis de rendre cette technologie accessible aux agriculteurs, quel que soit leur niveau technique. Enfin, les méthodologies de tests et de déploiement ont garanti la fiabilité et la robustesse du système en conditions réelles.

L'impact du projet Agriculture Cameroun se manifeste à plusieurs niveaux transformateurs. La démocratisation de l'expertise agricole permet désormais à tous les producteurs, des petits exploitants aux grandes fermes, d'accéder à des connaissances avancées traditionnellement réservées aux spécialistes. L'optimisation des ressources se traduit par une utilisation plus efficace et durable de l'eau, des engrais et des pesticides, réduisant ainsi les coûts et l'impact environnemental. La prévention des risques, grâce à l'anticipation des maladies des plantes et des conditions météorologiques défavorables, permet aux agriculteurs de prendre des décisions proactives plutôt que réactives. L'analyse économique intégrée aide à maximiser la rentabilité des exploitations en tenant compte des fluctuations du marché et des coûts de production.

\section*{Perspectives d'évolution}

L'évolution du système Agriculture Cameroun s'inscrit dans une vision progressive et ambitieuse, structurée autour de trois horizons temporels complémentaires.

À court terme, dans les six à douze prochains mois, nos efforts se concentreront sur l'enrichissement substantiel de la base de connaissances par l'intégration de nouvelles variétés de cultures spécifiquement adaptées aux différentes zones agro-écologiques du Cameroun. L'amélioration de l'interface utilisateur constituera également une priorité majeure, notamment par l'implémentation d'un support multilingue couvrant le français, l'anglais et les principales langues locales pour garantir une accessibilité maximale. L'optimisation des performances techniques sera poursuivie activement pour réduire les temps de réponse et améliorer la scalabilité du système face à une adoption croissante. L'intégration progressive de l'Internet des Objets (IoT) par la connexion avec des capteurs terrain permettra l'acquisition de données en temps réel, enrichissant considérablement la précision des analyses et recommandations.

Le développement à moyen terme, s'étalant sur une période de un à trois ans, marquera une montée en sophistication technologique significative. L'intégration d'algorithmes d'intelligence artificielle avancés, incluant des modèles de machine learning spécialisés en agriculture, permettra des analyses prédictives plus fines et des recommandations personnalisées. Le développement d'un système de recommandation intelligent adaptera automatiquement les conseils selon l'historique agricole et le profil spécifique de chaque utilisateur. La création d'une plateforme collaborative transformera le système en véritable réseau social d'agriculteurs, facilitant le partage d'expériences, de bonnes pratiques et de solutions innovantes entre pairs. Un module de formation interactif sera développé pour créer un écosystème d'apprentissage continu, combinant théorie agricole et pratiques adaptées au contexte local.

La vision à long terme, projetée sur trois à cinq ans, ambitionne une transformation radicale de l'agriculture dans la région. L'extension géographique du système vers d'autres pays d'Afrique centrale créera un réseau d'intelligence agricole transnational, favorisant les échanges de connaissances et l'harmonisation des pratiques. L'intégration de la technologie blockchain révolutionnera la traçabilité des produits agricoles, garantissant l'authenticité et facilitant la certification biologique et équitable. Le développement de capacités de prédiction climatique avancée, basées sur des modèles météorologiques sophistiqués, permettra aux agriculteurs de s'adapter proactivement aux défis du changement climatique. Enfin, la création d'un écosystème agricole complet intégrera harmonieusement les systèmes bancaires, d'assurance et de commerce, offrant aux agriculteurs un environnement numérique unifié pour la gestion globale de leur activité.

\section*{Ressources pour approfondir}

Pour poursuivre votre montée en compétence dans le domaine des systèmes multi-agents et de l'intelligence artificielle appliquée à l'agriculture, plusieurs voies d'apprentissage s'offrent à vous, chacune répondant à des besoins spécifiques de développement professionnel.

La formation continue constitue le socle fondamental de cette démarche d'approfondissement. Les plateformes d'apprentissage en ligne proposent des cours de référence, notamment le cours "Multi-Agent Systems" de l'University of Edinburgh sur Coursera, qui couvre les aspects théoriques avancés des SMA. Pour l'application spécifique à l'agriculture, le cours "Artificial Intelligence in Agriculture" de Wageningen University sur edX offre une perspective complète sur l'intégration de l'IA dans les pratiques agricoles modernes. Le programme "AI for Trading" d'Udacity, bien qu'orienté finance, fournit des concepts transposables à l'analyse des marchés agricoles et à la prédiction des prix des commodités.

L'obtention de certifications professionnelles renforce significativement votre crédibilité technique. La certification Google Cloud Professional Data Engineer valide votre expertise dans la gestion de pipelines de données complexes, compétence essentielle pour traiter les volumes importants d'informations agricoles. La certification AWS Certified Machine Learning - Specialty démontre votre maîtrise des outils d'apprentissage automatique dans l'écosystème Amazon, particulièrement utile pour le déploiement d'applications à grande échelle. La certification Microsoft Azure AI Engineer Associate complète cette panoplie en couvrant les aspects d'intégration d'intelligence artificielle dans des environnements d'entreprise.

L'engagement dans les communautés et événements professionnels enrichit considérablement votre réseau et vos connaissances. Les conférences internationales de référence incluent AAMAS (International Conference on Autonomous Agents and Multiagent Systems), qui présente les dernières avancées académiques et industrielles, ICAART (International Conference on Agents and Artificial Intelligence) pour une perspective plus large sur l'IA, et PRECISION AG (Precision Agriculture Conference) pour les applications spécifiques au secteur agricole. Les communautés en ligne offrent un accès quotidien à l'expertise collective : Stack Overflow avec ses tags spécialisés multi-agent-systems et google-adk, les forums Reddit r/MachineLearning, r/artificial et r/agriculture pour les discussions thématiques, et GitHub pour l'exploration de projets open-source en agriculture intelligente.

La veille technologique régulière vous permet de rester à la pointe des innovations. Les blogs spécialisés constituent des sources d'information privilégiées : le Google AI Blog (ai.googleblog.com) pour les dernières avancées de Google en IA, Towards Data Science (towardsdatascience.com) pour des articles techniques approfondis, et les sites d'innovations AgTech comme Precision Ag (www.precisionag.com) pour les applications concrètes en agriculture. Les newsletters et podcasts complètent cette veille : The Batch de deeplearning.ai pour une synthèse hebdomadaire des actualités IA, AI in Agriculture Podcast pour les discussions sectorielles, et Future of Food Podcast pour une vision prospective de l'agriculture de demain.

\include{Matter/RÉFÉRENCES }
\subsection{Annexe A : Glossaire des termes SMA}

\begin{description}
    \item[Agent] Entité autonome capable de percevoir son environnement et d'agir de manière indépendante pour atteindre ses objectifs.

    \item[Autonomie] Capacité d'un agent à prendre des décisions et à agir sans intervention externe directe.

    \item[Comportement (Behavior)] Ensemble d'actions et de réactions d'un agent face aux stimuli de son environnement.

    \item[Communication inter-agents] Mécanisme permettant aux agents d'échanger des informations et de coordonner leurs actions.

    \item[Coordination] Processus par lequel les agents synchronisent leurs actions pour atteindre un objectif commun.

    \item[Émergence] Phénomène par lequel des propriétés complexes apparaissent au niveau système à partir d'interactions simples entre agents.

    \item[Environnement] Contexte dans lequel évoluent les agents, incluant les ressources et les contraintes.

    \item[LLM (Large Language Model)] Modèle d'intelligence artificielle capable de comprendre et générer du langage naturel.

    \item[Multi-agent] Système composé de plusieurs agents interagissant dans un environnement partagé.

    \item[Ontologie] Représentation formelle des connaissances d'un domaine spécifique.

    \item[Performative] Type d'acte de communication dans le langage ACL (ex: INFORM, REQUEST, PROPOSE).

    \item[Pro-activité] Capacité d'un agent à prendre des initiatives et à anticiper les besoins.

    \item[Réactivité] Capacité d'un agent à répondre rapidement aux changements de son environnement.

    \item[Socialité] Capacité d'un agent à interagir et collaborer avec d'autres agents.

    \item[Système expert] Système d'intelligence artificielle simulant le raisonnement d'un expert humain dans un domaine spécifique.
\end{description}

\subsection{Annexe B : Code source complet des exemples}

\subsubsection{Structure du projet}
\begin{verbatim}
agriculture_cameroun/
├── src/
│   ├── agents/
│   │   ├── __init__.py
│   │   ├── agent_principal.py
│   │   ├── agent_meteo.py
│   │   ├── agent_cultures.py
│   │   ├── agent_sante_plantes.py
│   │   ├── agent_economique.py
│   │   └── agent_ressources.py
│   ├── tools/
│   │   ├── __init__.py
│   │   ├── weather_api.py
│   │   ├── plant_diagnosis.py
│   │   └── market_analysis.py
│   ├── config/
│   │   ├── __init__.py
│   │   ├── settings.py
│   │   └── ontology.py
│   └── main.py
├── tests/
│   ├── test_agents.py
│   └── test_integration.py
├── docs/
├── requirements.txt
├── pyproject.toml
└── README.md
\end{verbatim}

\subsubsection{Agent Principal - Code complet}
\begin{verbatim}
# src/agents/agent_principal.py
import asyncio
from typing import Dict, List, Any
from google_adk import Agent, tool, Context

class AgentPrincipal(Agent):
    """Agent coordinateur principal du système Agriculture Cameroun"""

    def __init__(self):
        super().__init__(
            name="CoordinateurAgricole",
            instructions="""
            Vous êtes le coordinateur principal d'un système d'aide
            aux agriculteurs camerounais. Votre rôle est de :
            1. Analyser les demandes des utilisateurs
            2. Router vers les agents spécialisés appropriés
            3. Synthétiser les réponses des sous-agents
            4. Fournir des recommandations cohérentes
            """,
            tools=[self.router_query, self.synthesize_responses]
        )
        self.sub_agents = {
            'meteo': AgentMeteo(),
            'cultures': AgentCultures(),
            'sante': AgentSantePlantes(),
            'economique': AgentEconomique(),
            'ressources': AgentRessources()
        }

    @tool
    async def router_query(self, query: str, context: Context) -> Dict[str, Any]:
        """Route la requête vers les agents appropriés"""

        # Analyse de la requête pour déterminer les agents concernés
        relevant_agents = self._analyze_query(query)

        results = {}

        # Envoi de la requête aux agents concernés
        for agent_name in relevant_agents:
            if agent_name in self.sub_agents:
                agent_response = await self.sub_agents[agent_name].process(
                    query, context
                )
                results[agent_name] = agent_response

        return results

    @tool
    async def synthesize_responses(
        self,
        agent_responses: Dict[str, Any],
        original_query: str
    ) -> str:
        """Synthétise les réponses des agents spécialisés"""

        synthesis = f"Réponse à votre demande : {original_query}\n\n"

        for agent_name, response in agent_responses.items():
            synthesis += f"**{agent_name.capitalize()}** :\n"
            synthesis += f"{response}\n\n"

        # Ajout de recommandations globales
        synthesis += self._generate_global_recommendations(agent_responses)

        return synthesis

    def _analyze_query(self, query: str) -> List[str]:
        """Analyse la requête pour identifier les agents concernés"""
        query_lower = query.lower()
        relevant_agents = []

        # Mots-clés pour chaque domaine
        keywords = {
            'meteo': ['météo', 'pluie', 'température', 'climat', 'saison'],
            'cultures': ['culture', 'plantation', 'variété', 'semis', 'récolte'],
            'sante': ['maladie', 'parasite', 'traitement', 'pesticide', 'fongicide'],
            'economique': ['prix', 'marché', 'vente', 'profit', 'coût'],
            'ressources': ['eau', 'irrigation', 'engrais', 'sol', 'fertilisant']
        }

        for agent, agent_keywords in keywords.items():
            if any(keyword in query_lower for keyword in agent_keywords):
                relevant_agents.append(agent)

        # Si aucun agent spécifique n'est identifié, consulter tous
        if not relevant_agents:
            relevant_agents = list(self.sub_agents.keys())

        return relevant_agents

    def _generate_global_recommendations(
        self,
        agent_responses: Dict[str, Any]
    ) -> str:
        """Génère des recommandations globales basées sur toutes les réponses"""

        recommendations = "**Recommandations générales** :\n"

        # Logique de synthèse intelligente
        if 'meteo' in agent_responses and 'cultures' in agent_responses:
            recommendations += "• Adaptez vos pratiques culturales aux prévisions météorologiques\n"

        if 'sante' in agent_responses and 'ressources' in agent_responses:
            recommendations += "• Coordonnez les traitements phytosanitaires avec la gestion de l'irrigation\n"

        if 'economique' in agent_responses:
            recommendations += "• Prenez en compte l'analyse économique pour optimiser votre rentabilité\n"

        recommendations += "• Consultez régulièrement le système pour des mises à jour\n"

        return recommendations

# Exemple d'utilisation
async def main():
    agent = AgentPrincipal()

    # Simulation d'une requête d'agriculteur
    requete = "Quand planter le maïs cette saison ? Il y a des taches sur mes feuilles de tomates."

    response = await agent.process(requete)
    print(response)

if __name__ == "__main__":
    asyncio.run(main())
\end{verbatim}

\subsubsection{Configuration et outils - Code complet}
\begin{verbatim}
# src/config/settings.py
import os
from typing import Dict, Any

class Settings:
    """Configuration globale du système"""

    # API Keys
    GOOGLE_API_KEY = os.getenv("GOOGLE_API_KEY")
    OPENWEATHER_API_KEY = os.getenv("OPENWEATHER_API_KEY")

    # Configuration des agents
    AGENT_CONFIG = {
        "model": "gemini-pro",
        "temperature": 0.7,
        "max_tokens": 1000
    }

    # Configuration météo
    WEATHER_CONFIG = {
        "default_location": "Yaoundé, CM",
        "forecast_days": 7,
        "update_interval": 3600  # 1 heure
    }

    # Configuration base de connaissances
    KNOWLEDGE_BASE = {
        "crops_db": "data/crops_cameroon.json",
        "diseases_db": "data/plant_diseases.json",
        "markets_db": "data/market_prices.json"
    }

    # Configuration logging
    LOGGING_CONFIG = {
        "level": "INFO",
        "format": "%(asctime)s - %(name)s - %(levelname)s - %(message)s",
        "file": "logs/agriculture_sma.log"
    }

# src/tools/weather_api.py
import aiohttp
import asyncio
from typing import Dict, Any, Optional
from config.settings import Settings

class WeatherAPI:
    """Interface pour l'API météorologique"""

    def __init__(self):
        self.api_key = Settings.OPENWEATHER_API_KEY
        self.base_url = "http://api.openweathermap.org/data/2.5"

    async def get_current_weather(self, location: str) -> Dict[str, Any]:
        """Récupère les conditions météorologiques actuelles"""

        url = f"{self.base_url}/weather"
        params = {
            "q": location,
            "appid": self.api_key,
            "units": "metric",
            "lang": "fr"
        }

        async with aiohttp.ClientSession() as session:
            async with session.get(url, params=params) as response:
                if response.status == 200:
                    data = await response.json()
                    return {
                        "temperature": data["main"]["temp"],
                        "humidity": data["main"]["humidity"],
                        "description": data["weather"][0]["description"],
                        "wind_speed": data["wind"]["speed"],
                        "pressure": data["main"]["pressure"]
                    }
                else:
                    raise Exception(f"Erreur API météo: {response.status}")

    async def get_forecast(self, location: str, days: int = 5) -> List[Dict[str, Any]]:
        """Récupère les prévisions météorologiques"""

        url = f"{self.base_url}/forecast"
        params = {
            "q": location,
            "appid": self.api_key,
            "units": "metric",
            "lang": "fr",
            "cnt": days * 8  # 8 prévisions par jour (toutes les 3h)
        }

        async with aiohttp.ClientSession() as session:
            async with session.get(url, params=params) as response:
                if response.status == 200:
                    data = await response.json()
                    forecasts = []

                    for item in data["list"]:
                        forecasts.append({
                            "datetime": item["dt_txt"],
                            "temperature": item["main"]["temp"],
                            "humidity": item["main"]["humidity"],
                            "description": item["weather"][0]["description"],
                            "precipitation": item.get("rain", {}).get("3h", 0)
                        })

                    return forecasts
                else:
                    raise Exception(f"Erreur API prévisions: {response.status}")

# src/tools/plant_diagnosis.py
import json
from typing import Dict, List, Any, Optional
from config.settings import Settings

class PlantDiagnosisEngine:
    """Moteur de diagnostic des maladies des plantes"""

    def __init__(self):
        self.diseases_db = self._load_diseases_database()
        self.symptoms_keywords = self._build_symptoms_index()

    def _load_diseases_database(self) -> Dict[str, Any]:
        """Charge la base de données des maladies"""

        try:
            with open(Settings.KNOWLEDGE_BASE["diseases_db"], 'r', encoding='utf-8') as f:
                return json.load(f)
        except FileNotFoundError:
            # Base de données par défaut
            return {
                "mildiou_tomate": {
                    "nom": "Mildiou de la tomate",
                    "symptomes": ["taches brunes", "feuilles jaunes", "pourriture fruits"],
                    "causes": ["humidité élevée", "température fraîche"],
                    "traitements": ["bouillie bordelaise", "aération", "espacement plants"],
                    "prevention": ["rotation cultures", "drainage", "traitement préventif"]
                },
                "fusariose": {
                    "nom": "Fusariose",
                    "symptomes": ["flétrissement", "jaunissement", "brunissement racines"],
                    "causes": ["champignon sol", "stress hydrique"],
                    "traitements": ["fongicide systémique", "amélioration drainage"],
                    "prevention": ["semences saines", "désinfection sol"]
                }
            }

    def _build_symptoms_index(self) -> Dict[str, List[str]]:
        """Construit un index des symptômes pour la recherche"""

        index = {}
        for disease_id, disease_info in self.diseases_db.items():
            for symptom in disease_info["symptomes"]:
                words = symptom.lower().split()
                for word in words:
                    if word not in index:
                        index[word] = []
                    if disease_id not in index[word]:
                        index[word].append(disease_id)

        return index

    def diagnose(self, symptoms_description: str) -> List[Dict[str, Any]]:
        """Diagnostique les maladies potentielles"""

        words = symptoms_description.lower().split()
        potential_diseases = {}

        # Recherche des maladies correspondantes
        for word in words:
            if word in self.symptoms_keywords:
                for disease_id in self.symptoms_keywords[word]:
                    if disease_id not in potential_diseases:
                        potential_diseases[disease_id] = 0
                    potential_diseases[disease_id] += 1

        # Tri par score de correspondance
        sorted_diseases = sorted(
            potential_diseases.items(),
            key=lambda x: x[1],
            reverse=True
        )

        # Formatage des résultats
        results = []
        for disease_id, score in sorted_diseases[:3]:  # Top 3
            disease_info = self.diseases_db[disease_id]
            results.append({
                "nom": disease_info["nom"],
                "probabilite": min(score * 0.3, 1.0),  # Score normalisé
                "symptomes": disease_info["symptomes"],
                "traitements": disease_info["traitements"],
                "prevention": disease_info["prevention"]
            })

        return results
\end{verbatim}

\subsection{Annexe C : Commandes utiles et dépannage}

\subsubsection{Installation et configuration}
\begin{verbatim}
# Installation de Python 3.12+
sudo apt update
sudo apt install python3.12 python3.12-pip

# Vérification de la version
python3.12 --version

# Installation de Poetry
curl -sSL https://install.python-poetry.org | python3 -

# Configuration du projet
poetry new agriculture_cameroun
cd agriculture_cameroun
poetry add google-adk aiohttp fastapi uvicorn

# Variables d'environnement
export GOOGLE_API_KEY="votre_clé_ici"
export OPENWEATHER_API_KEY="votre_clé_ici"

# Lancement du système
poetry run python src/main.py

# Tests
poetry run pytest tests/

# Interface web
poetry run uvicorn src.api:app --reload --port 8000
\end{verbatim}

\subsubsection{Debugging et logs}
\begin{verbatim}
# Activation du mode debug
export DEBUG=True

# Logs détaillés
export LOG_LEVEL=DEBUG

# Monitoring des performances
poetry add psutil
python -c "
import psutil
print(f'CPU: {psutil.cpu_percent()}%')
print(f'RAM: {psutil.virtual_memory().percent}%')
"

# Test de connectivité API
curl -X GET "http://api.openweathermap.org/data/2.5/weather?q=Yaoundé&appid=YOUR_KEY"

# Validation du format JSON
python -m json.tool data/crops_cameroon.json

# Profiling du code
poetry add py-spy
py-spy record -o profile.svg -- python src/main.py
\end{verbatim}

\subsubsection{Maintenance et mise à jour}
\begin{verbatim}
# Mise à jour des dépendances
poetry update

# Sauvegarde de la base de données
cp -r data/ backup/data_$(date +%Y%m%d_%H%M%S)/

# Nettoyage des logs
find logs/ -name "*.log" -mtime +30 -delete

# Test de santé du système
poetry run python tests/health_check.py

# Monitoring continu
watch -n 10 'curl -s http://localhost:8000/health | jq .'
\end{verbatim}

\subsection{Annexe D : FAQ et problèmes courants}

\subsubsection{Questions fréquentes}

\textbf{Q1 : Comment ajouter un nouvel agent spécialisé ?}

R : Pour ajouter un nouvel agent, suivez ces étapes :
\begin{enumerate}
    \item Créez une nouvelle classe héritant de \texttt{Agent}
    \item Définissez les outils spécifiques avec le décorateur \texttt{@tool}
    \item Ajoutez l'agent au dictionnaire \texttt{sub\_agents} de l'agent principal
    \item Mettez à jour la méthode \texttt{\_analyze\_query} avec les mots-clés appropriés
    \item Ajoutez des tests unitaires dans \texttt{tests/}
\end{enumerate}

\textbf{Q2 : Comment intégrer une nouvelle API externe ?}

R : Créez un nouveau module dans \texttt{src/tools/} avec :
\begin{itemize}
    \item Une classe d'interface API
    \item Gestion des erreurs et retry logic
    \item Cache pour optimiser les performances
    \item Tests de l'intégration
\end{itemize}

\textbf{Q3 : Comment personnaliser les recommandations par région ?}

R : Modifiez le fichier \texttt{config/regional\_settings.py} :
\begin{itemize}
    \item Ajoutez des données spécifiques par région
    \item Adaptez les algorithmes de recommandation
    \item Mettez à jour la base de connaissances régionale
\end{itemize}

\subsubsection{Problèmes courants et solutions}

\textbf{Erreur : "API Key non valide"}
\begin{itemize}
    \item Vérifiez que les clés API sont correctement définies dans les variables d'environnement
    \item Testez les clés avec un appel direct à l'API
    \item Régénérez les clés si nécessaire
\end{itemize}

\textbf{Erreur : "TimeoutError lors des appels API"}
\begin{itemize}
    \item Augmentez le timeout dans la configuration
    \item Implémentez un mécanisme de retry
    \item Utilisez un cache pour réduire les appels
\end{itemize}

\textbf{Performance lente du système}
\begin{itemize}
    \item Activez le cache Redis pour les réponses fréquentes
    \item Optimisez les requêtes à la base de données
    \item Utilisez le parallélisme pour les appels aux sous-agents
\end{itemize}

\textbf{Erreur : "Module non trouvé"}
\begin{itemize}
    \item Vérifiez que Poetry est correctement installé
    \item Exécutez \texttt{poetry install} pour installer les dépendances
    \item Activez l'environnement virtuel avec \texttt{poetry shell}
\end{itemize}

\textbf{Interface web ne se lance pas}
\begin{itemize}
    \item Vérifiez que le port 8000 n'est pas déjà utilisé
    \item Contrôlez les logs pour identifier l'erreur
    \item Testez avec un port différent : \texttt{--port 8001}
\end{itemize}

\subsubsection{Conseils de performance}

\begin{itemize}
    \item \textbf{Cache intelligent} : Implémentez un cache hiérarchique pour les données météo et market
    \item \textbf{Parallélisation} : Utilisez \texttt{asyncio.gather()} pour les appels simultanés aux agents
    \item \textbf{Pagination} : Limitez le nombre de résultats retournés par les APIs
    \item \textbf{Compression} : Compressez les réponses HTTP avec gzip
    \item \textbf{Monitoring} : Utilisez des métriques pour identifier les goulots d'étranglement
\end{itemize}

\subsubsection{Ressources de support}

\begin{itemize}
    \item \textbf{Documentation officielle} : \url{https://github.com/agriculture-cameroun/sma-docs}
    \item \textbf{Issues GitHub} : \url{https://github.com/agriculture-cameroun/sma/issues}
    \item \textbf{Email support} : --
\end{itemize}



% Bibliography.
\printbibliography




\end{document}
